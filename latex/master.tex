\documentclass[11pt, a4paper]{article}

%\usepackage{ngerman}
\usepackage[english]{babel}
\usepackage[utf8]{inputenc} %Korrekte Kodierung der Umlaute nach UTF-8
\usepackage[T1]{fontenc} %Korrekte Kodierung der Umlaute nach UTF-8
\usepackage{makeidx} %Zur automatischen Indexerstellung
\usepackage{amsfonts}
\usepackage{amssymb}
\usepackage{color}    % Verwendung von Farben
\usepackage{listings} % Korrekter Satz von Listings und Quellcode
\usepackage{tikz}     % graphen malen in latex
\usepackage{graphicx} % abbildungen einfügen
\usepackage{subfigure}
\usepackage{amsmath}  % mathe
\usepackage{amsthm}   % theoreme
\usepackage{footnote}
\usepackage{etoolbox} % common tools
\usepackage[colorlinks,
pdfpagelabels,
pdfstartview = FitH,
bookmarksopen = true,
bookmarksnumbered = true,
linkcolor = black,
urlcolor = black,
plainpages = false,
hypertexnames = false,
citecolor = black] {hyperref}
\usepackage{url}      % Korrekter Satz von URLs
% tkiz libs
\usetikzlibrary{calc,positioning,arrows,automata,decorations.pathreplacing,angles,quotes}


%Hilfs-Fonts - ohne Serifen (hier für Tabellen)
%\newfont{\bib}{cmss8 scaled 1040}
%\newfont{\bibf}{cmssbx8 scaled 1040}

% definitionen
\newtheorem{defi}{Definition}
\newtheorem{theoreme}{Theorem}
\newtheorem{axiom}{Axiom}
\newcommand{\code}[1]{\texttt{#1}}

\definecolor{lightgray}{gray}{0.85}
\setlength{\emergencystretch}{1em} % erlaube zusätzliche abstände

\definecolor{codegreen}{rgb}{0,0.6,0}
\definecolor{codegray}{rgb}{0.5,0.5,0.5}
\definecolor{codepurple}{rgb}{0.58,0,0.82}
\definecolor{backcolour}{rgb}{0.95,0.95,0.92}

% define coq style
\lstdefinestyle{coq}{
  morekeywords={Section, Module, End, Require, Import, Export,
    Variable, Variables, Parameter, Parameters, Axiom, Hypothesis,
    Hypotheses, Notation, Local, Tactic, Reserved, Scope, Open, Close,
    Bind, Delimit, Definition, Let, Ltac, Fixpoint, CoFixpoint, Add,
    Morphism, Relation, Implicit, Arguments, Unset, Contextual,
    Strict, Prenex, Implicits, Inductive, CoInductive, Record,
    Structure, Canonical, Coercion, Context, Class, Global, Instance,
    Program, Infix, Theorem, Lemma, Corollary, Proposition, Fact,
    Remark, Example, Proof, Goal, Save, Qed, Defined, Hint, Resolve,
    Rewrite, View, Search, Show, Print, Printing, All, Eval, Check,
    Projections, inside, outside, Def,
    forall, exists, exists2, fun, fix, cofix, struct,
    match, with, end, as, in, return, let, if, is, then, else, for, of,
    nosimpl, when,
    Type, Prop, Set, true, false, option,
    pose, set, move, case, elim, apply, clear, hnf,
    intro, intros, generalize, rename, pattern, after, destruct,
    induction, using, refine, inversion, injection, rewrite, congr,
    unlock, compute, ring, field, fourier, replace, fold, unfold,
    change, cutrewrite, simpl, have, suff, wlog, suffices, without,
    loss, nat_norm, assert, cut, trivial, revert, bool_congr, nat_congr,
    symmetry, transitivity, auto, split, left, right, autorewrite,
    by, done, exact, reflexivity, tauto, romega, omega,
    assumption, solve, contradiction, discriminate,
    do, last, first, try, idtac, repeat},
  morecomment=[s]{(*}{*)},
  morestring=[b]",
  morestring=[d]’,
  literate=
  {\\forall}{{\color{dkgreen}{$\forall\;$}}}1
  {\\exists}{{$\exists\;$}}1
  {<-}{{$\leftarrow\;$}}1
  {=>}{{$\Rightarrow\;$}}1
  {==}{{\code{==}\;}}1
  {==>}{{\code{==>}\;}}1
  % {:>}{{\code{:>}\;}}1
  {->}{{$\rightarrow\;$}}1
  {<->}{{$\leftrightarrow\;$}}1
  {<==}{{$\leq\;$}}1
  {\#}{{$^\star$}}1 
  {\\o}{{$\circ\;$}}1 
  {\@}{{$\cdot$}}1 
  {\/\\}{{$\wedge\;$}}1
  {\\\/}{{$\vee\;$}}1
  {++}{{\code{++}}}1
  {~}{{\ }}1
  {\@\@}{{$@$}}1
  {\\mapsto}{{$\mapsto\;$}}1
  {\\hline}{{\rule{\linewidth}{0.5pt}}}1,
  backgroundcolor=\color{backcolour}, 
  commentstyle=\color{codegreen},
  keywordstyle=\color{magenta},
  numberstyle=\tiny\color{codegray},
  stringstyle=\color{codepurple},
  basicstyle=\ttfamily\footnotesize,
  breakatwhitespace=false,         
  breaklines=true,                 
  captionpos=b,                    
  keepspaces=true,                 
  numbers=left,                    
  numbersep=5pt,                  
  showspaces=false,                
  showstringspaces=false,
  showtabs=false,                  
  tabsize=2
}

%Seitenformat-Definitionen
\topmargin0mm
\textwidth147mm
\textheight214mm
\evensidemargin5mm
\oddsidemargin5mm
\footskip19mm
\parindent=0in

% \renewcommand{\lstlistingname}{Auflistung}
\apptocmd{\thebibliography}{\raggedright}{}{} % bibliography right aligned
\newcounter{sectionnumber}
\newcommand*\rot{\rotatebox{90}}

\makeindex % legt das Index-File an

\begin{document}          

\begin{titlepage}
  \begin{center} 
    \mbox{}
    
    {\large \sc Masterthesis} \\    

    \vspace{1cm}
  
    {\huge Implementing the Raft consensus protocol with Velisarios in Coq\\[1em] {\LARGE}}  
        
    \vspace{2cm}
    
    \includegraphics[scale=0.05]{images/Mathnatlogo.jpg}\\[1em]
    University of Potsdam\\
    Institute for Computer Science\\
    Professorship for Theoretical Computer Science
    
    \vspace{2cm}
    
		submitted by
		
    \vspace{1em}
    
		{\Large Henrik Jürges} \\
        {Matr.-Nr. 751237}\\

    \vspace{2em}
        {Problem definition and supervision:}\\
        {Prof. Christoph Kreitz}\\
		    {Dr. Vincent Rahli}\\
    \vspace{3em}    
    Potsdam\\
    30. March 2020
  \end{center}
\end{titlepage}


\pagenumbering{gobble}
% Zweite Seite = Kurzzusammenfassung
\begin{center}
{\bf Abstract}
\end{center}
\noindent

\newpage

% Dritte Seite = Inhaltsverzeichnis
\tableofcontents 
\newpage

\pagenumbering{Roman}
% Abkürzungsverzichnis
\newpage
\begin{appendix}
%\section{Abkürzungen und Akronyme}\index{Akronyme}
\label{akro}
\begin{tabbing}
\hspace*{3cm}\=  \\ \kill
A2KB     \> Annotate to knowledge base\\
ASCII    \> American Standard Code for Information Interchange\\
C2KB     \> Concept to knowledge base\\
D2KB     \> Disambiguate to knowledge base\\
FTP      \> File Transfer Protocol\\
HITS     \> Hypertext-Induced Topic Search\\
HTML     \> Hypertext Markup Language\\
HTTP     \> Hypertext Transfer Protocol\\
IATA     \> International Air Transport Association\\
IE       \> Informationsextraktion\\
IEC      \> International Electrotechnical Commission\\
IRI      \> Internationalized Resource Identifiers\\
ISO      \> International Organization for Standardization\\
LOD      \> Linked Open Data\\
NED      \> Named Entity Disambigutation\\
NEL      \> Named Entity Linking\\
NEN      \> Named Entity Normalisation\\
NER      \> Named Entity Recognition\\
NIL      \> Not in List\\
NLP      \> Natural Language Processing\\
RDF      \> Resource Description Framework\\
RDFS     \> Resource Description Framework Schema\\
OWL      \> OWL 2 Web Ontology Language\\
SPARQL   \> SPARQL Protocol and RDF Query Language\\
URI      \> Uniform Resource Identifier\\
URL      \> Uniform Resource Locator\\
W3C      \> World Wide Web Community\\
WWW      \> World Wide Web\\
XML      \> Extensible Markup Language\\
\end{tabbing}
\newpage

\section{Präfixdefinitionen}\index{Präfix}
\label{präfix}

Im Verlauf dieser Arbeit werden verschiedene Internationalized Resource Identifiers in abgekürzter Form benutzt. 
Im Folgenden sind diese aufgezählt:
\begin{tabbing}
\hspace*{3cm}\=  \\ \kill
\verb|dbo:| \> \url{<http://dbpedia.org/ontology/>}\\
\verb|dbr:| \> \url{<http://dbpedia.org/resource/>}\\
\verb|rdf:| \> \url{<http://www.w3.org/1999/02/22-rdf-syntax-ns#>}\\
\verb|rdfs:| \> \url{<http://www.w3.org/2000/01/rdf-schema#>}\\
\verb|owl:| \> \url{<http://www.w3.org/2002/07/owl#>}\\
\verb|foaf:| \> \url{<http://xmlns.com/foaf/0.1/>}
\end{tabbing}
\newpage

\addcontentsline{toc}{section}{C~\,List of Figures}
%\addcontentsline{toc}{section}{~~~~Abbildungsverzeichnis} % Zeile für das Inhaltsverzeichnis
\listoffigures

%\newpage
\addcontentsline{toc}{section}{D~\,List of Tables}
%\addcontentsline{toc}{section}{~~~~Tabellenverzeichnis} % Zeile für das Inhaltsverzeichnis
\listoftables
\end{appendix}
\newpage

\pagenumbering{arabic}
\setcounter{page}{1}
\setcounter{section}{0}
\renewcommand{\thesection}{\arabic{section}}
\newpage

\section{Einleitung}
\label{sec_einleitung}

% Motivation

 
\newpage
%
\section{Aufbau und Inhalt der wissenschaftlichen Arbeit}
\label{sec_aufbau}

\newpage
%
\section{Velisarios}
\label{sec_3}



%%% Local Variables:
%%% mode: latex
%%% TeX-master: "../master"
%%% End:


\newpage
%
\section{Inhaltliche Bestandteile der Seminararbeit}
\label{sec_inhalt}

\newpage
%
\section{Evaluation}
\label{sec_5}

Because this thesis serves as a demonstration and
reference for future programmers a strong evaluation
based on benchmarking and comparison was not done.
A major reason was that only one other protocol
was already implemented with Velisarios which
has a greater implementation scope, for instance log compaction.
Also, other implementations of Raft didn't match this
state of implementation which makes it wrong to
compare them. This only leads to false assumptions
and conclusions about Raft. So, the evaluation
is only a recapitulation of the implementation process.

The goal of this thesis was to implement the Raft protocol
with Velisarios. It should show that the programmers
effort to implement concensus protocols with COQ and
Velisarios are less error-prone and easier than other
approaches. Velisarios stands on a solid
logical foundation, the logic of events, and the
abstractions introduced by it narrow the effort
left for the implementors of distributed systems.
Additionally, COQ requires the programmer to rethink
its approach by providing only deterministic functions
and clean algebraic types. These restrictions and
the strong type-checking done by COQ lead to short and
precise functions. The functions are very predictive as they
act as mathematical functions with idempotence, so
a function returns the same outputs on the same inputs.

The strong distinction between state and functions operating
on that state which are only dependent types, lead to a
precise description of the protocol to implement.
The only failures that can happen are logical ones
in the COQ side of the code. The OCaml side is a little
bit different. Also, OCaml is a strong-typed language
it allows to ignore the type-checker and to mutate
types into different ones. This feature is heavily
used by the glue code to bridge the gap between
the COQ and OCaml type systems. This interface
can lead to all sorts of errors. These errors need
to be carefully traced because they originate in either
the COQ side or the OCaml side or both. So, an implementor
should write test cases for the main parts of the protocol
and verify the correctness of these. Maybe a verification
of the COQ code with Velisarios \code{EventOrderings}
can wipe out all failures on this side. But this part
was out of the reach of this thesis. 

Some parts of the description of Raft are not
obviously as expected. For example, the linearizable
semantics are a great concept and is described in a
detailed fashion but some points are left unsanswered.
The linearizable semantics are a key point to
prevent the global state machine to process requests
twice by using sessions for clients which are replicated
in the log. It uses a cache to respond to client requests
already processed but the concrete way of organizing
(either in the log or otherwise) and distributing of
the cache across the network are left to the readers
imagination. A more precise description of such things
can lead to a more comparable code base across the
different implementations since every different approach
has a great impact on the types and amount of messages
used. The core Raft only postulates four types
a messages but taken the case above into account a fifth
type is needed or not.

At a last point the tooling on the COQ side and the OCaml
side is not great. The COQ import facilities need a lot
of hand-craft to find the dependencies and work different
in the editor and the command-line. Upgrading the COQ
version can lead to broken builds and the need to adapt
the code to this specific version. On the OCaml side
the tooling is much better but requires the programmer
to craft and keep care of build files by themself.
A more elaborated tooling can substantial improve
the working and collaboration between the COQ and OCaml
ecosystems.

%%% Local Variables:
%%% mode: latex
%%% TeX-master: "../master"
%%% End:

\newpage
%
\section{Conclusion and Outlook}
\label{sec_conclusion}


\subsection{Conclusion}

This thesis showed the implementation of Raft using
Velisarios and the Logic of Events as a fundamental
basis for distributed consensus protocols. The Raft
thesis describes the protocol quite detailed and
straight foreward which makes the implementation
goes along with the description. Also, many existing
implementations\footnote{\url{https://raft.github.io/}}
help to correctly interpret some not very clear points.
The diffcult by implementing something with COQ is that
it requires programmers from other languages or unfamiliar
with the functional concepts to rethink many parts which
are based on mutation in other languages. Also, the
restriction to algebraic types and deterministic recursive
functions is not forseen by the description of Raft.
But these led to a more predictable and less error-prone
code so that mostly only logical failures are left as
bugs in the code. As shown with this thesis Velisarios
is a perfect fit for implementing distributed protocols
and using them in the real world. But Velisarios is
addressed to programmers already familiar with the
concepts of \texttt{proofs-as-programs} and functional
programming in COQ. The basic aspects are described
in the paper but the understanding about how the
parts fit together in code are left to the example
implementation of PBFT. In conclusion, this thesis
may serve as an additional reference point for
further approaches and implmementions from non-theoretical
computer scientist.

\subsection{Outlook}

This thesis only showed the basic implementation of
Raft. There are a plethora of optimization points
and evaluation parts left. The original Raft paper
introduces membership changes in a cluster of nodes.
Since Velisarios does assumptions about the nodes
in the network beforehand and uses these assumptions
for the bijective representation of nodes and names
this feature can be quite tedious to implement.
Additionally, the Raft protocol provides log
compaction like most of the other Paxos protocols.
This can be easily integrated into the existing
implemntation by introducing a bunch of new message
types. With this feature the Raft protocol can be
compared to other protocols like the PBFT variant
implemented by the original Velisarios authors.
Also, the proving of the safety properties of
Raft is not done with the event orderings
introduced by Velisarios. With the proofs
done the implemented version of Raft
can be seen as a fully \texttt{proof-as-program}.
On a larger scale, more approaches to implement and
logical describe complex systems are needed. The
complexity of these system are hard to design
correctly by hand-woven code in non-functional
languages. So, these approaches maybe a common
standard in the future to handle complex systems
and make the life of programmers easier.

%%% Local Variables:
%%% mode: latex
%%% TeX-master: "../master"
%%% End:



\newpage
% \begin{appendix}
% \setcounter{section}{4}
% \renewcommand{\thesection}{\Alph{section}}
% \input{chapters/appendix.tex}
% \end{appendix}

%Hier kommt das Literaturverzeichnis
\newpage
\addcontentsline{toc}{section}{~~~~Bibliography} % Zeile für das Inhaltsverzeichnis
\bibliography{bibfile}
\bibliographystyle{plain}

\newpage
%Hierhin kommt der Index (Sachverzeichnis)
%\addcontentsline{toc}{section}{Index} % Dies ist die Zeile für das Inhaltsverzeichnis
%\flushbottom                                    
%\printindex

\newpage
\section*{Selbstständigkeitserklärung}

Hiermit erkläre ich, Henrik Jürges, an Eides statt, dass ich die vorliegende Arbeit
"`Erweiterung des GERBIL-Frameworks zur Evaluation von Named-Entity-Linking Verfahren"'
selbstständig angefertigt, nicht anderweitig zu Prüfungszwecken vorgelegt und
keine anderen als die angegebenen Hilfsmittel verwendet habe. Sämtliche 
wissentlich verwendete Textausschnitte, Zitate oder Inhalte anderer Verfasser 
wurden ausdrücklich als solche gekennzeichnet und im Literaturverzeichnis aufgeführt.
\vspace{2mm}

Potsdam, den 29. April 2016
\vspace{4mm}

\newlength\us
\settowidth{\us}{-Henrik~Jürges-}
\begin{tabular}{p{\us}}\hline
\centering\footnotesize Henrik~Jürges
\end{tabular}

\end{document}

%%% Local Variables:
%%% mode: latex
%%% TeX-master: t
%%% End:
