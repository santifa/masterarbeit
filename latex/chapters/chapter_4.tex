%
\section{Raft and Velisarios}
\label{sec_4}

The last chapter described the Raft protocol and shows
only a brief overview of Velisarios. The overview was
short because this chapter goes into more detail
and it's easier to follow the real working of Velisarios
with an concrete example. So, this chapter combines both
technologies and presents an implementation of the Raft
protocol with the Velisarios framework.

This section will go into the details of implementing a protocol
with Velisarios. It shows the whole process of adjusting the Raft
protocol to fit into the Velisarios rational and the problems
along the way. 

The code is public available on
Github\footnote{\url{https://github.com/santifa/masterarbeit/tree/master/raft}}.
It contains Velisarios as a git submodule but due to the limitations how
the COQ importing system works the necessary files from the current
master branch are copied into the raft working directory. The files
are reordered to have clear distinction between files belonging to
the Velisarios core, useful definitions which are some sort of dependency
and the protocol files.

\subsection{Contextualisation}
As seen before, Velisarios is an abstract framework to model
a distributed system as a set of state machines which comminucate
with messages. A state machine in the context of Velisarios is
a process which runs the consensus protocol. The term process is used
to refer to the Velisarios state machine while the term state
machine refers to the abstract state machine in Raft. The state
machine is kept abstract since its real manifestation doens't care
by implementing Raft. Velisarios uses the term replica to refer
to some sort of server or node.

\begin{lstlisting}[style=coq,label=lst:abs,
caption=Abstract typeclasses used by Velisarios.]
Global Instance Raft_I_Msg : Msg := MkMsg RaftMsg.
Global Instance Raft_I_get_msg_status : MsgStatus := MkMsgStatus Raftmsg2status.
Global Instance Raft_I_Node : Node := MkNode RaftNode RaftnodeDeq.
Global Instance Raft_I_Quorum : Quorum_context := MkQuorumContext...
Global Instance TIME_I : Time.
Global Instance Raft_I_context : RaftContext := MkRaftContext...
\end{lstlisting}

Since Velisarios is build as an abstraction it provides typeclasses
for inserting real components for these abstract types. Listing~\ref{lst:abs}
shows the binding for the Raft components to the Velisarios typeclasses.
As one can see only basic definitions for a protocol are needed.
The heavy lifting happens by implementing the context which
gets passed at creation time of a node.

% Velisarios is an abstract framework which models individual
% state machines which communicate with messages over channels.
% This abstraction enables the implementer to choose its own
% types for the state machines and messages. But the types
% are defined beforehand.

\begin{lstlisting}[style=coq,label=lst:context,
caption=Excerpt of the context definition of types and functions used within
the protocol.]
   Content : Set;
   content2string : Content -> string;
   RaftSM : Set;
   RaftSM_result : Set;
   RaftSM_initial_state : RaftSM;
   RaftSM_update_state : RaftSM -> Content -> RaftSM * RaftSM_result
\end{lstlisting}

Listing~\ref{lst:context} shows the abstraction done beforehand. The
implementation starts with an definition which parts are kept abstract
until it gets realized with a concrete programm or state machine and
a concrete log. The \code{Content} describes an abstract type for the log
cells. By this definition a log can only have one sort of types for all cells.
The concrete state machine is the \code{RaftSM} type. To update the state
machine \code{RaftSM\_update\_state} is defined which takes the current
state of the program and some log cell as input. It produces a next
state and some sort of result (\code{RaftSM\_result}). Although the abstract
types for replicas and clients are defined in this excerpt which are left out. 

This is defined as dependent type record in COQ which enables to use the
abstract definitions as their were implemented. 


% COQ provides dependent types as \code{Class} with an explict constructor.
% In listing~\ref{lst:context} the excerpt of context for the Raft protocol is presented.
% The definition of replicas and clients are left out. It's important to note
% that the defined types and functions are only abstract and can be instantiated
% afterwards. This leads to the circumstances that parts of the protocol can 
% be abstract too but used within the implementation as if they're already
% implemented. \code{Content} refers to the type of the log in the Raft protocol
% and \code{RaftSM} is the type of the state machine. An update via
% \code{RaftSM\_update\_state} leads to a new state machine and some result of
% type \code{RaftSM\_result} and depends on the type of the log and the type of
% state machine itself.

Velisarios uses an abstract definition of nodes which can run state machines.
Therefore, one defines the different types inductively as in
listing~\ref{lst:nodes}. Raft uses only one type of server nodes, the
\code{replica} as the state transitions are handled by the process itself.
The \code{client} is only mentioned for completness as the real implmenetation
is done within OCaml. One has to proof that the choosen type has decidable
equaltity. The last line bounds the definitions to the abstract node defined
by Velisarios. Thus, the name of the nodes can be linked to the corresponding
process. 

\begin{lstlisting}[style=coq,label=lst:nodes,
caption=The different kinds of raft nodes.]
  Inductive RaftNode :=
    | replica (n : Rep)
    | client (n : Client).

  Global Instance Raft_I_Node : Node := MkNode RaftNode RaftnodeDeq.
\end{lstlisting}

The linking is done by an bijective mapping between the node names which are
ordered natural numbers and the provided type definition. This is used by
Velisarios quorum theory to ensure that the majority of nodes agree
un updating its process state on the same operation.~\cite{rahli2018velisarios}

\begin{defi}
  Give a set of nodes $N$ than a byzantine quorum is a subset $Q\subseteq N$ with the
  property $f + 1\leq (2*|Q|)-(|N|)$. Every two choosen quorums should intersect,
  such that the number of nodes are greater than a possible number of failed
  nodes.
\end{defi}


% As mentioned in the last chapter, Velisarios uses names to identifies nodes
% within the network. 
% In listing~\ref{lst:nodes} are the kind of nodes defined. Only two nodes are
% distinguished by the system whereby the real client implementation is done
% in Ocaml. The function \code{replica2rep} takes a node and returns the unique
% name of the node in the network which is analogous for clients. This is done
% by matching over the different kinds of the type \code{RaftNode}. Although,
% one has to show that the node type has an decidable equality before it
% can be used by Velisarios as a real instance for the networks nodes.
% The last line shows how the real definition of \code{RaftNode} is bound
% as a type parameter to the abstract typeclass for nodes provided by Velisarios.

% maybe state the quorum integration


\subsection{Basic Definitions}
After the last subsection, Velisarios knows about the node types and it's proven
that the quorum theory holds. The abstract types for state machine and the log
for the Raft protocol are already defined. This subsection defines the
terminology used by Raft in the whole protocol.

\paragraph{Terms and time}
Raft divides time into abstract chunks. These terms spanning from the
voting for a new leader til the possible failure of a leader.

\begin{lstlisting}[style=coq,label=lst:terms,
caption=The definition of terms and timer.]
Inductive Term := term (n : nat).
Inductive Timer := timer (id : nat).
\end{lstlisting}

The terms are modeled as monotonallicaly increasing natural numbers to
distinguish between two terms. The current term is stored by every node
participating in the network and is exchanged with every request to
inform nodes which might run in older terms. Although, the term is used
to to achive the leader completeness property.
Raft uses a timeout mechanic to detect failed leaders or split
votes. This is modeled in Velisarios as an internal message which
indicates a timeout. Therefore, every node has an internal counter
which stores the last send timer message. On arrival of such a message
the node checks if the timer number matches with the internal one. If
that's the case the node starts a state transition or drops the message
otherwise.

\paragraph{State transistions}
Raft uses three different states for a node to model
the different states of the network.

\begin{lstlisting}[style=coq,label=lst:states,caption=The definition of the nodes states.]
 Record LeaderState :=
    Build_Leader_State
      {
        next_index : NextIndex;
        match_index : MatchIndex;
      }.

  Record CandidateState :=
    Build_Candidate_State { votes : nat; }.

  Inductive NodeState :=
  | follower
  | candidate (c : CandidateState)
  | leader (l : LeaderState).
\end{lstlisting}

As shown in listing~\ref{lst:states} a node can be in follower state which
doesn't need to store additional informations. The follower state is 
the initial state of a node when the network starts. A follower switches to the
candidate state if no leader is present in the network which it detects
using timeout messages. It stores the amount of votes for itself. 
Since every candidate votes for itself the \code{votes} variable is initialized 
with 1. If a candidate wins the election it transistions to the leader state. 

The leader manages two internal lists of indeces. The \code{NextIndex} is the
list of the next possible log index the leader can send to the node. 
It is initialized to the leaders next log index. For instance, 
if the leader recognizes that one node is behind with log replication then
the leader decrements the \code{NextIndex} and tries to replicate the lower
log index. The \code{MatchIndex} is a list that stores the highest known replicated
log index for every node. This enables a leader to check if it can commit a
log index which is replicated by the majority of the network.
The full process states for a raft node are presented in the next section along
with the updating functions.

\paragraph{Log}
One of the essential parts of the Raft protocol is the coherent log which gets
replicated across the network of nodes. Raft uses the log for different aspects.
It stores the replicated client requests as well as sessions or possible other
things like end of a term. The log can be easaly extended by further messages
which must be stored and replicated.

In listing~\ref{lst:log} the definition of the log is presented. The
\code{EntryType} enables to differentiate between protocol internals
and the clients \code{Content}. At the moment, there are only session and
end of term entries allowed beside the normal log content. Every log entry
stores its creation term along with the message. For simplicity a log 
is only a different notation for a list of entries.

\begin{lstlisting}[style=coq,label=lst:log,caption=The definition of raft log.]
  Inductive EntryType :=
  | content (c : Content)
  | session (s : Sessions)
  | new_term.

  Record Entry :=
    MkEntry
      {
        entry_term : Term;
        entry : EntryType;
      }.
  Notation Log := (list Entry).
\end{lstlisting}

The \code{new\_term} is slightly misleading because it marks the start of
a new term in the log which can only be appended by a leader. Since a
term starts with the voting round one or more voting rounds can happen
and the next entry maybe 3 terms ahead of the last \code{new\_term} entry.
Sessions are explained in the next paragraph.

% To implement the Raft protocol various basic definitions should
% be done. These definitions declare the semantics used in the later
% protocol and ease the implementation.

\paragraph{Linearizable semantics}
Raft uses the linearizable semantic to prevent the network from
processing the same request twice. The issue of processing client
request in the correct order and only once applies to all
concesus protocols and mostly handled the same way.
The clients first message is a registeration on the network.
This creates a session which is stored in the log and gets
replicated across the network.

\begin{lstlisting}[style=coq,label=lst:linearizable,
caption=The definitions used to implement the linearizable semantics for raft.]
  Inductive SessionId := session_id (n : nat).
  Inductive RequestId := request_id (n : nat).
  Notation Sessions := (list (SessionId * Client)).

  Record Cache :=
    Build_Cache
      {
        sid : SessionId;
        rid : RequestId;
        result : option RaftSM_result;
      }.
\end{lstlisting}

The client generates request numbers that along with the session id identify
new requests to handle. As shown in listing~\ref{lst:linearizable} sessions
and requests are natural numbers which are incremented.  The sessions are kept as
long as the network is alive and some function to unregister a client
is not provided. The nodes are keeping an internal cache of processed
requests for all clients. If a client request is detached to the network
it gets stored in the cache with the result \code{None}. This means
the request is dispatched and waiting to be completed. If the leader
has a result for this request it stores it alongside the requests. 
This prevents network from processing some request twice.
This is not an ideal solution since it enforces more book-keeping on every nodes
side. Although, the Raft protocol allows read-only messages for faster
processing but these are not implemented in this version. 

\paragraph{Messages}
Velisarios uses messages that are directly send over a channel to communicate
between nodes. A process reacts to incoming messages with handler methods that
possibly update the process state and create outgoing messages. Velisarios
distinguishes between three different message types. Input from clients
and other nodes are set as external messages. If a replica sends a message
to itself its declared as internal one and all messages belonging to the
protocol are stated as ones. The implementation is shown in listing~\ref{lst:msgstate}.

\begin{lstlisting}[style=coq,label=lst:msgstate,
caption=Messages have different state informations which are handled by Velisarios.]
  Inductive msg_status :=
  | MSG_STATUS_PROTOCOL
  | MSG_STATUS_INTERNAL
  | MSG_STATUS_EXTERNAL.
\end{lstlisting}

Raft uses only a few messages for the core functionality. Extended functionality
like log compaction or cluster membership changes introduce more messages for
the communication. The overview of used messages is shown in listing~\ref{list:msgs}.

\begin{lstlisting}[style=coq,label=lst:msgs,
caption=Overview of the different messages needed for the core Raft.]
 Inductive RaftMsg :=
  | init_msg (offset: nat)
  | register_msg (register : RegisterClient)
  | register_response_msg (result : Result)
  | request_msg (request : ClientRequest)
  | response_msg (result : Result)
  | append_entries_msg (entries : AppendEntries)
  | append_entries_response_msg (result : Result)
  | request_vote_msg (vote : RequestVote)
  | request_vote_response_msg (result : Result)
  | timer_msg (timer : Timer).
\end{lstlisting}

The listing is straight foreward. If the system boots up it gets
an initial message which defines a timer offset for every node. 
A client which wants to participate in the network sends a register
message with its name to the possible leader. It gets a response
which indicates if the registration was successfull and its session.
If the registration fails because the node wasn't the leader it
gets a hint to the current leader. 
To send a request the client sends its name, session and some request id
along with the command to the leader. Afterwards the leader issues an 
\code{AppendEntries} message to the network. Every node responds to
the call with either success or failure. For checks if some node
is outdated every protocol message carries the term number. 
If the system needs a new leader the followers transistions to
candidate state and issues request vote messages to the rest of the
network. Every node responds to this message with a successfull vote
if the node doesn't voted for some other node nor itself.
The timer message is used to simulate the internal timing of the nodes.

In listing~\ref{lst:append} are the two different types of \code{AppendEntries}
message shown. Raft uses the message either for a heartbeat mechanism or for
the log replication. In Raft the leader issues heartbeats which are basically
empty replicate messages to demonstrate that leader is still alive. The
replication variation additionally carries the log entry and the session and
request from the client. This is done to distinguish responses from the nodes
to the correct client. 

\begin{lstlisting}[style=coq,label=lst:append,
caption=The two possible append entry calls in Raft.]
Inductive AppendEntries :=
 | heartbeat (term : Term) (leader : Rep) (last_log_index : nat)
             (last_log_term : nat) (commit_index : nat)
 | replicate (term : Term) (leader : Rep) (last_log_index : nat)
             (last_log_term : Term) (commit_index : nat)
             (entry : list Entry) (id : SessionId * RequestId).
\end{lstlisting}

The different types of result messages are combined into a single \code{Result}
type. The listing~\ref{lst:result} shows the inductive type. For each
message which could lead to a response a reslt value is added to the type.
Since Raft uses only a few types of messages only four types are needed.
The listing is straight forward. The first one is the result for a client
registering at the network. It either succeds with a session or fails
with a hint to the correct leader. The next one is the response to
a voting request which indicates if the vote was granted by the requested
node. The append entries result sends the name of the follower and the request and
session id to the leader. This is done to indicate the overall replication state
in the match and next index. The session and request id are usefull to
recognize the correct client request and prevent double execution.

\begin{lstlisting}[style=coq,label=lst:result,
caption=The inductive result type for all response messages.]
Inductive Result :=
  | register_client_res (status: bool) (session_id : SessionId) (leader : option Rep).
  | request_vote_res (term : Term) (vote_granted : bool)
  | client_res (status : bool) (result : RaftSM_result)
  | append_entries_res (term : Term) (success : bool) (node : Rep) (id : SessionId * RequestId)
\end{lstlisting}

These are the most basic definitions needed for the Raft protocol. The other
definitions are left out because they're either straight foreward or the
implementation details don't give a deeper understanding.

\subsection{Raft middleware}
This subsection shows the details of the core Raft implementation.
After defining the abstract context and the terms used in Raft
now the process states and transition or handler functions are
described.

\paragraph{State}
Listing~\ref{lst:state} shows the definition of the Raft process
state. Each Raft process needs to know the current term and for
which node it voted at the beginning of the term. Also, it stores
the current term leader and the log. The commit index indicates which
log entry was last applied to the global state machine. This
may be different to the log index this process has at last applied
to its state machine. Each process stores its current \code{NodeState}
which could change through transitioning into the next term.
To fullfill the property that each node should have a different
timeout a base timeout is set at network bootup.
The \code{timer} indicates the last timer message number the node
reacts to. And the \code{msg\_timer} is used to resend
failed messages if the node is in leader mode.

\begin{lstlisting}[style=coq,label=lst:state,
caption=The dependent record definition of the Raft process state.]
Record RaftState :=
  Mk_State
    {
      current_term : Term;
      voted_for : option name;
      leader_id : option name;
      log : Log;
      commit_index : nat;
      last_applied : nat;
      sm : RaftSM;
      node_state : NodeState;
      timeout : nat;
      timer : Timer;
      msg_timer : Timer;
    }. 
\end{lstlisting}

These are the most important values and others are left out
for a better overview. Since COQ doesn't provide a short
syntax for updating dependent type records a dependency is used
which abuses the typeclass
mechanics.\footnote{\url{https://github.com/tchajed/coq-record-update}}
This leads to cleaner code for the process transistions.

Only one process transistion is mentioned here because the others are
only changing one value.
The more complex one happens if a node changes it terms by advancing
to the next one. The current term gets incremented and the current
leader as well as the voting informations are resetted.
Additionally it returns to the follower state.
The code is shown in listing~\ref{lst:advance}.
If the node transistions to the candidate because it
recognizes a failed leader this is done in other definitions
which uses this general one.

\begin{lstlisting}[style=coq,label=lst:advance,
caption=The process transition to advance to the next term.]
Definition advance_term (s : RaftState) (t : Term) :RaftState :=
  if (current_term s) <? t then
    s <| current_term := t |> <| voted_for := None|>
      <|leader_id := None |> <| node_state := follower |>
  else s.
\end{lstlisting}

The \code{<| ... |>} syntax is the update notation provided by the
record update mechanics. It works by converting a record to
an applicative where every field is an identity function which
can be replaced by an update function provided by the elements
surrounded by \code{<| |>}.

\paragraph{Node rules}
The nodes need basic rules to work properly and apply the
properties proposed by Raft. These rules are implemented alongside
the update function either by a separate definition or inside
of an update function.

The most basic rule ist when to apply a new log entry.
Raft defines that an log entry is applied when the commit index
is higher than the last applied index. Listing~\ref{lst:apply} shows
the implementation on how to apply a log index. 

\begin{lstlisting}[style=coq,label=lst:apply,
caption=How a log is applied to the local state machine.]
Definition apply_to_sm (s : RaftState) : (RaftState * option RaftSM_result) :=
  if (last_applied s) <? (commit_index s) then 
   let s' := increment_last_applied s in 
   let e := get_log_entry (log s') (last_applied s') in
   match e with
    | None => (s, None) 
    | Some e' => 
      match (entry e') with
        | content e'' => 
          let (sm, result) := RaftSM_update_state (sm s) e'' in
          let s'' := update_sm s sm in
          (s'', Some result)
        | _ => (s, None) 
      end
   end
  else (s, None). 
\end{lstlisting}

First it is checked if the commit index is higher and if so the last
applied index is incremented. The corresponding log entry is recievied.
If there is an entry and it's a content entry than apply it to the
state machine and return the result and new state. 
If the entry is not a content entry return the new
state with the incremented last applied index or the old state otherwise.

Since the log can have multiple types of entries which are replicated
across the network the commit index may point to an entry which
is used for internal processing and not the next content entry.
Only one entry is applied at a time. Because this definition is run
every time the nodes recieve append entry messages backlogs are kept up
in idle times of the network.

The transitioning between the node states follow the rules:
\begin{itemize}
  \item Return to follower if a message with a higher term is recieved.
  \item Become a candidate if you're not a leader and you recieve a valid
    timeout message.
  \item Become leader if you're a candidate and you get the majority
    of votes.
\end{itemize}

The transistions are shown in listing~\ref{lst:trans1}. There is a general
function updating the node state. If a node transistions to a candidate
it advances to the next term. By becoming the leader the last index of
the log is retrieved to initialize the next index which is the next possible
empty index of the leaders log. The match index (not shown here) is
set to 0 and increases monotonallicaly when the leader learns the
log length of its followers. 

\begin{lstlisting}[style=coq,label=lst:trans1,
caption=The definitions on how a node changes it node state.]
Definition to_follower (s : RaftState) :=
update_node_state s follower.

Definition to_candidate (s : RaftState) :=
let s' := increment_term s in
update_node_state s' (candidate candidate_state0).

Definition to_leader (s : RaftState) (slf : Rep) : RaftState :=
let lli := get_last_log_index (log s) in
let l := new_leader (lli + 1) slf in 
update_node_state s (leader l).
\end{lstlisting}

The other rules are implemented along-side the update function since
they're not general enough or there are exceptions from some general rule.

\paragraph{Update functions}
As mentioned in the last section the update functions are the core
of Velisarios. For every message type an appropriate function is implemented
which handles this type of message. It changes the process state and
creates respond messages. To keep the functions more self-contained the
message creation is mostly done by some corresponding \code{mk\_foo\_msg}
function.

One of the most crucial parts for a node is the timeout.
Raft declares that a node has an internal timer to measure
if the leader is byzantine or a election failed. 
Since Velisarios and COQ has no support
for timers in the process state another way is used to model this part.
After initialization a node starts to send itself messages with a
certain delay. It tracks the number of this timer message.
If a message from the leader arrives it sends a new timer
message with a higher number. Listing~\ref{lst:timer} shows
the code to handle recieved timer messages. If the timer message
number is not equal to the stored one the message is dropped.
Otherwise, the node checks its own node state.
If the node is the leader it restarts the timer and send heartbeats
to the other nodes. The leader uses a shorter timeout than other nodes
to keep itself established as the leader.
If the node is some other node it reacts to the timer message by
starting a new election. This advances the term and the node 
transitions to candidate state.

\begin{lstlisting}[style=coq,label=lst:timer,
caption=The implementation of handling timer messages.]
  Definition handle_timer_msg (slf : Rep) : Update RaftState Timer DirectedMsgs :=
    fun state msg =>
      if TimerDeq (timer state) msg then
          match node_state state with
          | leader l =>
            let (s, timer) := mk_leader_timer state slf in
            (Some s, [mk_heartbeat_msg s slf; timer])
          | _ =>
            let (s, msg) := start_election state slf in
            let (s', timer) := mk_timer s slf in
            (Some s', [msg; timer])
          end
      else (Some state, []).
\end{lstlisting}

After startup or if a leader becomes byzantine one of the nodes timer messages
is recieved by that node and the node starts an election.
Listing~\ref{lst:vote} shows the update function for an election.
To start an election the node transistions to candidate state, votes for
itself and send request vote messages to all other nodes in the network.

A node recieving a vote request it restarts its internal timer. This prevents
all nodes from becoming candidates. Afterwards, the request is evaluated by the
following rules.
\begin{itemize}
  \item Ignore old terms
  \item Return to follower mode if the vote term is greater
  \item Check if the candidates log is at least up-date with the nodes log
\end{itemize}

If these checks succed than the node grants the vote if it already voted
for the candidate or it has not yet voted. Otherwise, the request is
rejected.  

\begin{lstlisting}[style=coq,label=lst:vote,
caption=The definitions on how a node changes it node state.]
  Definition handle_request_vote (slf : Rep) : Update RaftState RequestVote DirectedMsgs :=
    fun state msg =>
      let (s, timer) := mk_timer state slf in
      match msg with
      | request_vote t candi lli llt =>
        if TermLt t (current_term state) then
          (Some state, [mk_vote_response_msg (current_term state) false candi])
        else
          let s' := equal_term_or_follower s t in 
          if andb (if leader_id s' then false else true)
                  (check_entry_term (log s') lli llt ) then
            match voted_for s' with
            | None =>
              let s'' := update_voted_for s' (Some (replica candi)) in
              (Some s'', [mk_vote_response_msg (current_term state) true candi; timer])
            | Some (replica c) =>
              let valid := if rep_deq c candi then true else false in
              (Some s', [mk_vote_response_msg (current_term state) valid candi; timer])
            | _ => (Some s', [mk_vote_response_msg (current_term state) false candi])
            end
          else (Some s, [mk_vote_response_msg (current_term state) false candi; timer])
      end.
\end{lstlisting}

The candidate recieves the responses from it vote request and restarts its
timer. It checks if the vote was granted or if not its own term is outdated.
If the vote was granted it add this vote to the others and checks if has a
majority. The node wins if it hast at least more than 50 percent of
the possible votes. The \code{try\_to\_become} in listing~\ref{lst:rvote}
leader function calculates this, transistions the node to leader state and
issues heartbeat messages if so or leaving the state as is.

\begin{lstlisting}[style=coq,label=lst:rvote,
caption=The definitions on how a node changes it node state.]
  Definition handle_response_vote (slf : Rep) : Update RaftState Result DirectedMsgs :=
    fun state msg =>
      match msg with
      | request_vote_res t g =>
        if g then 
          let s := add_vote state in
          let (s', msgs) := try_to_become_leader s slf in
          if is_leader s' then
                 let (s'', timer) := mk_leader_timer s' slf in
                 (Some s'', msgs ++ [timer])
          else
            let (s'', timer) := mk_timer s' slf in
            (Some s'', msgs ++ [timer])
        else
          let s := equal_term_or_follower state t in
          (Some s, [])
      | _ => 
        (Some state, [])
      end.
\end{lstlisting}

After a certain amount of election rounds a leader arises from the
network. Now the system is ready to serve client registrations and requests.

A client has to register at the network in order to send requests to it.
It sends a \code{RegisterClient} message to some node with its
name as indication. If the node is not the leader it returns
false, the default session and a hint to the current leader.
If there is no current leader it returns none indicating that
the network is busy with a new election.

\begin{lstlisting}[style=coq,label=lst:register,
caption=The definitions on how a node changes it node state.]
  Definition handle_register (slf : Rep) : Update RaftState RegisterClient DirectedMsgs :=
    fun state msg =>
        match msg with
        | register_client c =>
          match (node_state state) with
          | leader _ =>
            let (s, id) := mk_session state c in
            let result := register_client_res true id (Some slf) in
            let broadcast := mk_broadcast_session_msg slf s in
            (Some s, mk_register_response_msg c result ++ broadcast)
          | _ =>
            let result := register_client_res false session_id0 (leader_id state) in
            (Some state, mk_register_response_msg c result)
         end
       end.
\end{lstlisting}

If the node is the leader it creates a new session for this client and
returns true, the session id for that client and also a hint to itself.
The leader appends the session to the log and starts replicating it
over the network.

After the client is registered at the network and knows the current leader
it can send requests messages. Therfore, it sends a \code{ClientRequest}
containing the its session id, a request id and the command to replicate
to the leader. If the node is not the leader it returns false and a hint
to the current leader. A candidate returns false without a hint to
the current leader. The leader checks if the request was already processed
and returns the result if it was cached. If the request is a new one the
leader appends it to its own log and starts sending append entry messages
to all other nodes in the network. The update function is shown in listing~\ref{lst:request}.

\begin{lstlisting}[style=coq,label=lst:request,
caption=The definitions on how a node changes it node state.]
  Definition handle_request (slf : Rep) : Update RaftState ClientRequest DirectedMsgs :=
    fun state r =>
      match node_state state with
      | candidate _ => (Some state, [])
      | follower =>
        let msg := mk_forward_msg state (request_msg r) in
        (Some state, msg)
      | leader _ =>
        match r with
        | client_request c si ri cmd =>
          if valid_session (sessions state) si then
            match in_cache (cache state) si ri with
            | Some res =>
              let msg := cached_request res c in
              (Some state, msg)
            | None =>
              let (s, entry) := add2log state cmd in
              let dmsgs := mk_append_entries_msg slf s entry (si, ri) in
              let s' := incr_leaders_next_index s in
              (Some s', dmsgs)
            end
          else (Some state, [])
        end
      end.
\end{lstlisting}

To replicate the log over the complete network the leader uses
append entries messages. A node recieving such a messages checks its
state and the message type. The node only reacts if it's not the leader node.
If the message is a heartbeat one the node maybe updates its term and
state and resets the timer. 
If the message is a replcate one the node maybe updates its term, state
and the leader. Then it applies the following rules.

\begin{itemize}
  \item Return false if message term < node term
  \item return false if there is no matching previous log entry
  \item If the previous log entry is in conflict with the leaders one
    delete it and all follwing enries.
  \item Append any new entries
  \item if the commit index of the leader is greater than of the node
    set commit index to $min(leader commit index, index of last new entry)$
\end{itemize}

The rules are enforcing the log replication safety along with the
rules for leader election. If the logs don't match they're
are build from the last matching point. The process can be modeled
in multiple ways. In this thesis only the simple step-wise approach is
used. The leader incrementally tracks back to the last matching index
and then builds the new log from there. The update function for
handling append entries message is shown listing~\ref{lst:replicate}.


\begin{lstlisting}[style=coq,label=lst:replicate,
caption=The definitions on how a node changes it node state.]
  Definition handle_append_entries (slf : Rep) : Update RaftState AppendEntries DirectedMsgs :=
    fun state msg =>
      match node_state state, msg with
      | leader _, _ => (Some state, [])
      | _, heartbeat t l lli llt ci =>
        let s := update_term_and_leader state t l in
        let (s', timer) := mk_timer s slf in
        (Some s', [timer])

      | _, replicate t l lli llt ci entries id =>
         let s := update_term_and_leader state t l in
        if (TermLt t (current_term state)) then
          let result := append_entries_res (current_term state) false slf id in
          (Some state, [MkDMsg (append_entries_response_msg result) [replica l] 10])
        else
          if (negb (check_entry_term (log state) lli llt )) then
            let result := append_entries_res (current_term state) false slf id in
            (Some state, [MkDMsg (append_entries_response_msg result) [replica l] 20])
          else
            let new_log := append_entries2log entries lli t state in
            let s := (update_log state new_log) in
            let s' := handle_log_entry s ci in
            let result := append_entries_res (current_term s') true slf id in
            (Some s', [MkDMsg (append_entries_response_msg result) [replica l] 0])
      end.
\end{lstlisting}

The leader collects and handles the result messages for an every append entries
message. Additionally, the leader tracks send states for append entries calls.
If a node fails to answer in a certain amount of time the leader retries
the message until a result message is recieved.
The leader increases the next andd match index for the node if the result is
success. Afterwards, it checks if the log entry is replicated over a majority
of the network nodes and maybe applies the log entry to the state machine.
If this is the case it sends a result message to the client or waits for
other anwsers otherwise.
If the answer result if false the leader checks if its term is outdated and
takes the appropriate action.

\begin{lstlisting}[style=coq,label=lst:rreplicate,
caption=The definitions on how a node changes it node state.]
  Definition handle_append_entries_result (slf : Rep) : Update RaftState Result DirectedMsgs :=
    fun state msg =>
      match node_state state, msg with
      | leader l, append_entries_res t true node (si, ri) =>
        let ni := increase_next_index (next_index l) node in
        let l' := update_leader_state ni (match_index l) in
        let s := update_node_state state (leader l') in
        if is_replicated s then
          let s' := increment_commit_index s in
          let (s'', result) := apply_to_sm s' in
          match result with
          | None => (Some s, [])
          | Some r =>
            let res := client_res true r in
            let client_msg := mk_client_response_msg res  in
            (Some s'', client_msg)
          end
        else (Some s, [])
      | leader l, append_entries_res t false node id =>
        if TermLt (current_term state) t then
          let s := equal_term_or_follower state t in
          (Some s, [])
        else
          (Some state, [])
      | _, _ => (Some state, [])
      end.
\end{lstlisting}

The other update functions are not essential to the Raft protocol
and can be viewed in the repository. The last step to implement the
Raft protocol is to provide a general update function
which assign an update function to every message which could
be recieved. Afterwards, the initial state and the general
update function are provided to Velisarios as how to construct
a Raft process as shown in listing~\ref{lst:update}.

\begin{lstlisting}[style=coq,label=lst:update,
caption=The definitions on how a node changes it node state.]

  Definition replica_update (slf : Rep) : MUpdate RaftState :=
    fun state m =>
      match m with
      | init_msg d => handle_init_msg slf state d
      | register_msg d => handle_register slf state d
      | register_result_msg d => handle_register_result slf state d
      | broadcast_sessions_msg d => handle_sessions_broadcast slf state d
      | request_msg d => handle_request slf state d
      | response_msg d => handle_response slf state d
      | append_entries_msg d => handle_append_entries slf state d
      | append_entries_result_msg d => handle_append_entries_result slf state d
      | request_vote_msg d => handle_request_vote slf state d
      | response_vote_msg d => handle_response_vote slf state d
      | forward_msg msg => handle_forward_msg slf state msg
      | timer_msg d => handle_timer_msg slf state d
      | debug_msg d => handle_debug_msg slf state d
      end.

  Definition RaftReplicaSM (slf : Rep) : MStateMachine _ :=
    mkSM
      (replica_update slf)
      (state0).
\end{lstlisting}

\subsection{Using Raft from OCaml}
The last subsection shows how to implement the rules
from Raft in the logic and semantics of Velisarios.
This only provides an abstract implementation of
Raft. The abstract parts defined in the contextualisation
are left for the realization. This is done partial
in COQ and OCaml. In COQ one has to provide a real
defintion for the state machine and the update function
of this state machine. Additionally, the types for
a log entry and a transistion result are defined.

\begin{lstlisting}[style=coq,label=lst:example,
caption=The definitions on how a node changes it node state.]
  Definition smState : Set := nat.
  Definition result : Set := nat.
  Definition content : Set := nat.

  Definition update_sm (s : smState) (c : content) :=
    let s' := s + c in
    (s', s').

Definition local_replica :=
  @RaftReplicaSM (@Raft_I_context).

Extraction "RaftReplicaEx.ml" state2string lrun_sm RaftdummySM local_replica (* leader_replica *) DirectedMsgs2string name2string.
\end{lstlisting}

Listing~\ref{lst:example} shows an excerpt of this definition.
The types are used to create a context for the realization.
The last definition is the main function used from OCaml to
create new replicas.
The extraction line tells COQ that it should create a file
\code{RaftReplicaEx.ml} and extract all following definitions
and the dependencies needed to run these into this file. 
Alongside this file, COQ produces a corresponding \code{*.mli}
file which is an OCaml interface file and abstracts the
konkrete code away from the caller.

To use the extracted code a bunch of glue code is needed.
The simulator\footnote{\url{https://github.com/santifa/masterarbeit/tree/master/raft/simulator}}.
serves as glue code for the protocols.
The simulator is an abstraction written in OCaml to ease
real-world implementations using the protocol middlewares
written with Velisarios.  
For instance, it provides logging facilities, 
the possibility to generate RSA keys which are used the original
PBFT protocol and the simulator abstraction for
running the implementations. The last one provides
a convient handling for the replicas and provides
a mechanic to program the behavior of a client and
the processing of requests by replicas.

Since the abstraction takes care of the heavy lifting
the programmer only needs to provide the conversion of
messages to strings, how replicas are created and if they
need some sort of initial messages. The major parts left
are the behavior of the client and how the replicas process
messages from the input queue.

These can be easily adopted from the examples of the
other protocols. Also, a more concrete version of
the simulator is provided which can be used for
specific test cases, since it uses a general message
processing and only the client, replica creation and
message to string conversion are left to the programmer.

\vspace{2em}

This chapter demonstrated how the Raft protocol can be
easily implemented with Velisarios. The abstraction
enforced by Velisarios and COQ lead to a clean and
easy to follow code base with a clean semantic which
is quite near the original description of Raft.
It started with describing the abstraction needed
for the state machine which is run by Raft and
the basic terms by simple algebraic types and
definitions. Afterwards, the definitions bound
to Velisarios were presented. The main part
covered the update functions and the process
state for the Raft middleware. 



%%% Local Variables:
%%% mode: latex
%%% TeX-master: "../master"
%%% End:
