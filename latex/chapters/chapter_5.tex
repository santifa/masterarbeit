%
\section{Evaluation}
\label{sec_5}

The goal of this thesis was to implement the Raft protocol
with Velisarios. It should show that the programmers
effort to implement concensus protocols with COQ and
Velisarios are less error-prone and easier than other
approaches. Velisarios stands on a solid
logical foundation, the logic of events, and the
abstractions introduced by it narrow the effort
left for the implementors of distributed systems.
Additionally, COQ requires the programmer to rethink
its approach by providing only deterministic functions
and clean algebraic types. These restrictions and
the strong type-checking done by COQ lead to short and
precise functions. The functions are very predictive as they
act as mathematical functions with idempotence, so
a function returns the same outputs on the same inputs.

The strong distinction between state and functions operating
on that state which are only dependent types, lead to a
precise description of the protocol to implement.
The only failures that can happen are logical ones
in the COQ side of the code. The OCaml side is a little
bit different. Also, OCaml is a strong-typed language
it allows to ignore the type-checker and to mutate
types into different ones. This feature is heavily
used by the glue code to bridge the gap between
the COQ and OCaml type systems. This interface
can lead to all sorts of errors. These errors need
to be carefully traced because they originate in either
the COQ side or the OCaml side or both. So, an implementor
should write test cases for the main parts of the protocol
and verify the correctness of these. Maybe a verification
of the COQ code with Velisarios \code{EventOrderings}
can wipe out all failures on this side. But this part
was out of the reach of this thesis. 

Some parts of the description of Raft are not
obviously as expected. For example, the linearizable
semantics are a great concept and is described in a
detailed fashion but some points are left unsanswered.
The linearizable semantics are a key point to
prevent the global state machine to process requests
twice by using sessions for clients which are replicated
in the log. It uses a cache to respond to client requests
already processed but the concrete way of organizing
(either in the log or otherwise) and distributing of
the cache across the network are left to the readers
imagination. A more precise description of such things
can lead to a more comparable code base across the
different implementations since every different approach
has a great impact on the types and amount of messages
used. The core Raft only postulates four types
a messages but taken the case above into account a fifth
type is needed or not.

At a last point the tooling on the COQ side and the OCaml
side is not great. The COQ import facilities need a lot
of hand-craft to find the dependencies and work different
in the editor and the command-line. Upgrading the COQ
version can lead to broken builds and the need to adapt
the code to this specific version. On the OCaml side
the tooling is much better but requires the programmer
to craft and keep care of build files by themself.
A more elaborated tooling can substantial improve
the working and collaboration between the COQ and OCaml
ecosystems.

Because this thesis serves as a demonstration and
reference for future programmers a strong evaluation
based on benchmarking and comparison was not done.
A major reason was that only one other protocol
was already implemented with Velisarios which
has a greater implementation scope, for instance log compaction.
Also, other implementations of Raft didn't match this
state of implementation which makes it wrong to
compare them. This only leads to false assumptions
and conclusions about Raft. So, the evaluation
is only a recapitulation of the implementation process.


%%% Local Variables:
%%% mode: latex
%%% TeX-master: "../master"
%%% End:
