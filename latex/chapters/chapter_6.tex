%
\section{Conclusion and Outlook}
\label{sec_conclusion}


\subsection{Conclusion}

This thesis showed the implementation of Raft using
Velisarios and the Logic of Events as a fundamental
basis for distributed consensus protocols. The Raft
thesis describes the protocol quite detailed and
straight foreward which makes the implementation
goes along with the description. Also, many existing
implementations\footnote{\url{https://raft.github.io/}}
help to correctly interpret some not very clear points.
The diffcult by implementing something with COQ is that
it requires programmers from other languages or unfamiliar
with the functional concepts to rethink many parts which
are based on mutation in other languages. Also, the
restriction to algebraic types and deterministic recursive
functions is not forseen by the description of Raft.
But these led to a more predictable and less error-prone
code so that mostly only logical failures are left as
bugs in the code. As shown with this thesis Velisarios
is a perfect fit for implementing distributed protocols
and using them in the real world. But Velisarios is
addressed to programmers already familiar with the
concepts of \texttt{proofs-as-programs} and functional
programming in COQ. The basic aspects are described
in the paper but the understanding about how the
parts fit together in code are left to the example
implementation of PBFT. In conclusion, this thesis
may serve as an additional reference point for
further approaches and implmementions from non-theoretical
computer scientist.

\subsection{Outlook}

This thesis only showed the basic implementation of
Raft. There are a plethora of optimization points
and evaluation parts left. The original Raft paper
introduces membership changes in a cluster of nodes.
Since Velisarios does assumptions about the nodes
in the network beforehand and uses these assumptions
for the bijective representation of nodes and names
this feature can be quite tedious to implement.
Additionally, the Raft protocol provides log
compaction like most of the other Paxos protocols.
This can be easily integrated into the existing
implemntation by introducing a bunch of new message
types. With this feature the Raft protocol can be
compared to other protocols like the PBFT variant
implemented by the original Velisarios authors.
Also, the proving of the safety properties of
Raft is not done with the event orderings
introduced by Velisarios. With the proofs
done the implemented version of Raft
can be seen as a fully \texttt{proof-as-program}.
On a larger scale, more approaches to implement and
logical describe complex systems are needed. The
complexity of these system are hard to design
correctly by hand-woven code in non-functional
languages. So, these approaches maybe a common
standard in the future to handle complex systems
and make the life of programmers easier.

%%% Local Variables:
%%% mode: latex
%%% TeX-master: "../master"
%%% End:

