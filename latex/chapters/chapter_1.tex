
\section{Introduction}
\label{sec_1}

% Motivation
\begin{quote}
A distributed system is one in which the failure of a computer you didn't even
know existed can render your own computer unusable. - Leslie Lamport
\end{quote}

In der modernen Welt sind Computer allgegenwärtig und mit den Entwicklungen
der letzten 20 Jahre haben Netzwerke von Computern eine immer wichtigere Rolle
für die Entwicklung der Menschheit gespielt. Sie ermöglichen physikalische
Limitierung der Ressourcen eines Computers aufzuweichen durch Bündelung
mehrerer Computer zu einem Virtuellen. Verteilte Systeme die sich einem Nutzer
als ein koheräntes System darstellen, sind in vielen Anwendungsfällen essential
für moderne Softwareanwendungen geworden, zum Beispiel Datenbanken die
auf mehrere Computer verteilt sind. Solche verteilten Systeme ermöglichen
es das Ausfälle einzelner Teilnehmer keinen Einfluss auf das Gesamtsystem haben,
solange noch genügend Teilnehmer verbleiben. Solche Ausfälle können nicht nur
durch dne Ausfall eines Computer entstehen, auch Netzwerkverbindungen können
ausfallen oder extrem langsam sein. Damit Informationen zwischen den
Instanzen gleich sind wurden Protokolle entwickelt, die auf die speziellen
Gegebenheiten angepasst sind.

Solche Protokolle sind bekanntermaßen schwer zu implementieren und stellen
die Entwickler vor die Herausforderung wie man sie testet. Da sie darauf
basieren, dass Teile eines verteilten Systems unerwartet ausfallen können,
ist die Anzahl der Möglichen Fehler extrem Groß und schwer in definierten
Umgebungen zu replizieren. Auch die Verifikation gestaltet sich als
schwierig, da viele Interaktionen statt finden und keine zeitliche
Kausalität angenommen werden kann.

Mit Hilfe von Theorembeweisern und formalen Methoden der Theoretischen
Informatik ist es Möglich Teile der Mathematik zu formalisieren und
diese zur Verifikation von formalisierten Algorithmen zu verwenden.
Die ``Logic of Events'' ist eine mathematische Theorie, um Protokolle
in verteilten Systemen zu beschreiben.
Die Theorie wurde in der Programmiersprache EventML umgesetzt, welches eine
ML-Dialekt ist. sie besitzt eine Schnittstelle zu NUPRL, einem Theorembeweiser,
um Eigenschaften verteilter Protokolle formal zu beweisen.
Außerdem ist es mit EventML möglich ausführbare Programme zu erzeugen und
diese auch in Simulationen zu testen und genauer zu untersuchen.

%Forschungsfragen:
In dieser Arbeit werden die folgenden Forschungsfragen erörtert:
\begin{itemize}
  \item 
\end{itemize}

%Struktur:

Im nächsten Abschnitt werden die Grundlagen für die ``Logic of Events'' nach
Bickford hergeleitet und diese Eingeführt. Danach wird EventML
als Programmiersprache genauer vorgestellt. In Abschnitt~\ref{sec_3} wird
der Prozess von EventML zu asuführbaren Code genauer betrachtet. Danach wird gezeigt wie
NUPRL Eigenschaften von verteilten Protokollen beweisen kann. Die Simulation
von in EventML geschriebenen Systemen wird im Abschnitt~\ref{sec_5} gezeigt.
Im letzten Abschnitt wird eine Zusammenfassung und ein Ausblick gegeben.
