\section{Glossar} %Appendix (Glossar)

Hier sollten die wichtigsten Schlüsselbegriffe, die in Ihrer Ausarbeitung thematisiert werden, knapp -- also wie in einem Lexikon -- gesammelt an einer Stelle zentral erläutert werden.
Das Glossar sollte im Umfang nicht mehr als eine Seite Ihrer Ausarbeitung einnehmen.
Sie müssen also allgemein bekannte Begriffe, wie z.B. \glqq Internet\grqq\, nicht unbedingt hier mit aufnehmen.

Die Erläuterung eines Begriffes im Glossar schließt nicht aus, dass Sie diesen Begriff bei seiner Einführung im Text nicht auch bereits erklärt haben. 


\begin{description}
\item[Browser:]\index{Browser} Ein spezielles Programm, mit dem man über das WWW Zugang zu WWW-Servern erlangen und von diesem angeforderte Dokumente anzeigen kann.

\item[Client:]\index{Client} Bezeichnet ein Programm, dass einen Server kontaktiert und von diesem Informationen anfordert. Der im WWW eingesetzte Browser ist in diesem Sinne ein Client. Aber es gibt auch andere Clients im WWW, die WWW-Server kontaktieren und Informationen von diesen herunterladen, wie z.B.~Suchmaschinen oder Agenten.

\item[HTML:]\index{HTML} Hypertext Markup Language; das einheitliche Dokumentenformat für Hyper\-media-Dokumente im WWW. Dokumente, die im WWW übertragen und vom Browser dargestellt werden sollen, sind in HTML kodiert.

\item[HTTP:]\index{HTTP} Hypertext Transfer Protocol; das Protokoll, das die Kommunikation von Browsern und WWW-Servern im WWW regelt. Fordert ein Browser ein Dokument vom WWW-Server an oder beantwortet der WWW-Server eine Anfrage, muss diese Anfrage den Konventionen des HTTP-Protokolls gehorchen. 

\item[Netzanwendung:]\index{Netzanwendungen} Ein Anwendungsprogramm, dessen Ablauf den Zugriff auf Ressourcen einschließt, die nicht lokal auf dem ausführenden Rechner\index{Rechner} liegen, sondern auf einem entfernten Rechner über das Netzwerk zugegriffen werden. 

\item[Server:]\index{Server} Bezeichnet einen Prozess\index{Prozess}, der von Clients kontaktiert wird, um diesen Informationen zurück zu liefern.
Oft wird auch der Rechner, auf dem ein Server-Prozess abläuft, als Server bezeichnet.
\end{description}


\newpage

% Appendix (Akronyme)
\section{Abkürzungen und Akronyme}\index{Akronyme}
Hier sollten {\bf alle} von Ihnen im Text verwendeten Abkürzungen in einem Verzeichnis zusammengestellt werden. 
Falls im Text keine Abkürzungen benutzt werden, brauchen Sie natürlich auch kein Abkürzungsverzeichnis zu erstellen.
\begin{tabbing}
\hspace*{3cm}\=  \\ \kill
4CIF \> 4 fach Common Intermediate Format\\
AAC \> Advanced Audio Coding\\
AAL \> ATM Adaption Layer\\
ABR \> Available Bit Rate\\
AC \> Audio Code\\
ACK \> Acknowledgement \\
ADM \> Add Drop Multiplexer\\
ADSL \> Asymmetric Digital Subscriber Line\\
AH \> Authentication Header\\
AIFF \> Audio Interchange File Format\\
AM \> Amplituden-Modulation\\
ANSI\> American National Standards Institute\\
API \> Application Programming Interface\\
ARP \> Address Resolution Protocol \\
W3C \> World Wide Web Community\\
WWW \> World Wide Web\\
\end{tabbing}
