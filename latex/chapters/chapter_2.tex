%
\section{Eventlogik}
\label{sec_logik}

%% einleitung %%
In diesem Kapitel wird die Eventlogik~\cite{bickford2003logic} nach Bickford und
Constable vorgestellt. Dazu werden die benötigten grundlegenden Logiken
und Theorien eingeführt.

%% grundlagen %%
\subsection{grundlegende Logiken}
% TODO
% Aussagenlogik
% Prädikatenlogik
% Logik 1. Ordnung
% dynamische Logik?
% Automatentheorie NEA

%% eventlogik %%
\subsection{Eventlogik}
Die Eventlogik nach Bickford und Constable baut auf der intuitunistischen Logik
auf. Sie verfolgt den ``correct-by-construction'' Ansatz, nach dem ein
funktionales Programm aus einem konstruktiven Beweis extrahiert werden kann.
Die Beweise haben die Grundform $\forall x:A.\exists : B.spec(x,y)$. Der konstruktive Beweis
liefert einen Extraktterm, der eine Realisierung des Beweises darstellt.
Da die Eventlogik die konstruktive Typentheorie benutzt sind alle
beweisbaren logischen Aussagen auch programmierbar, in dem Sinne, dass sie einen
Extraktterm besitzen der die Spezifikation erfüllt.~\cite{bickford2003logic}

% nachrichtenautomat
Grundlage für die Eventlogik ist ein Nachrichtenautomat. Dieser ist ein
nichtdeterministischer Automat der auf Nachrichten operiert.
Er hat die drei Basisoperationen ``senden'', ``empfangen'' und ``interne
Zustandsänderung''.~\cite{bickford2003logic}

\begin{defi}
  Ein Event $e=(a+m)$ ist ein Typ $A+M$ der entweder eine interne Aktion $A$ enthält
  oder eine empfangene Nachrichten $M$.
\end{defi}

\begin{defi}
  Ein Nachrichtenautomat ist ein abhängiger Datentyp mit der Struktur:
  $\{St, Act, Msg: Type, init: St$
    $f:(Act+Msg)\rightarrow St\rightarrow St$
    $send:(Act+Msg)\rightarrow St\rightarrow MsgList\}$  
\end{defi}

% verteilte Nachrichtenautomaten
Um verteilte Systeme darstellen zu können wird mehr als ein Nachrichtenautomat
benötigt, da ein Nachrichtenautomatenohne Interaktionen ein normaler
nichtdeterministischer Automat ist. Ein verteiltes System $V$ ist eine Menge von
Nachrichtenautomaten die in einem Netzwerk miteinander verknüpft sind.
Das Netzwerk ist fair, d.h. jede gesendete Nachricht wird auch unverändert
empfangen.~\cite{bickford2003logic}

\begin{defi}
  Ein verteiltes System $V$ ist ein Tupel $V=(M,L)$ wobei $m_i\in M$ mit $i\in
  \mathbb{N}$ die Menge der teilnehmenden Nachritenautomaten ist. $L$ beschreibt
  ein Kante in einem gerichteten Graphen und besteht aus dem Tupel $L=(s,d,m_i,m_j,Q)$ 
  mit $i,j\in \mathbb{N} und i \uneq j$. $Q$ ist eine Nachrichtenqueue.
\end{defi}




