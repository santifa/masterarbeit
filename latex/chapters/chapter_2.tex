%
\section{Aufbau und Inhalt der wissenschaftlichen Arbeit}
\label{sec_aufbau}

Im vorangegangenen Kapitel hatten wir bereits die Gliederung einer wissenschaftlichen Arbeit im Fachgebiet der Informatik kurz vorgestellt und erläutert, welche inhaltlichen Punkte in der \glqq Einleitung\grqq\, behandelt werden sollten.
Die folgenden Abschnitte skizzieren inhaltlich die übrigen der bereits genannten Gliederungspunkte.

\subsection{Verwandte Arbeiten und wissenschaftlicher Hintergrund (Related Work)}
%%
Hier sind vor allem zwei inhaltliche Punkte zu berücksichtigen:
\begin{itemize}
\item {\bf Notwendige Vorarbeiten und Grundlagen, die zum Verständnis der Arbeit notwendig sind}

Keine bzw. kaum eine Forschungsarbeit beginnt als \glqq tabula rasa\grqq , d.h. meist bauen wir auf  vorhandenen Grundlagen bzw. Vorarbeiten auf.
Die zum Verständnis der eigenen Forschungsarbeit notwendigen Grundlagen\index{Grundlagen} und Voraussetzungen müssen in diesem Kapitel skizziert bzw. zusammengefasst werden.
Dabei sollte man vom durchschnittlichen Kenntnisstand eines Informatikers ausgehen, d.h. Allgemeinplätze und allzu Grundlegendes hat hier nichts zu suchen.
Genauso soll hier nicht notwendigerweise eine kompletter Wissenschaftszweig in epischer Tiefe ausgebreitet werden, sondern lediglich die zum Verständnis notwendigen Teilbereiche in skizzenhafter Form und mit Angabe von Literaturhinweisen\index{Literaturhinweise} zusammengefasst werden.

\smallskip

\item {\bf Alternative Ansätze und Forschungsarbeiten zum Thema}

Besonders wichtig ist es, spezielle Vorarbeiten und alternative Forschungsansätze zum behandelten Thema darzulegen.
Dieser Abschnitt ist der von Ihnen durchgeführten Literaturrecherche gewidmet.
Welche Arbeiten zum aktuellen Thema gibt es? Wie sind andere Wissenschaftler an das Thema herangegangen? Hatten Sie Erfolg?

Wichtig ist, dass Sie jede der vorgestellten Arbeiten 
\begin{itemize}
\item korrekt zitieren (Bibliografie),
\item kurz die wichtigsten Ergebnisse bzw. Strategien skizzieren und
\item diese (kurz und knapp) in Zusammenhang mit ihrer eigenen Arbeit stellen. 
\end{itemize}
Wie unterscheidet sich der eigene Ansatz von den vorgestellten Arbeiten? 
Warum ist der eigene Ansatz eventuell erfolgsversprechender? 

\end{itemize}

\subsection{Eigener (wissenschaftlicher) Beitrag}
%%
Hier haben Sie die Freiheit, Ihren eigenen Arbeiten angemessen viel Raum zur Verfügung zu stellen.
Achten Sie dabei auf einen logischen Aufbau der Darstellung, d.h. Grundlegendes zuerst.
\begin{itemize}
\item Wie sind Sie vorgegangen?
\item Wo gibt es Probleme?
\item Wie werden diese gelöst?
\item Schreiben Sie in verständlicher Weise und drücken Sie sich dabei jeweils möglichst präzise, d.h. unmissverständlich aus (vgl. Kap.~\ref{sec_stil})
\item Verwenden Sie Abbildungen, Tabellen und Beispiele.
\item Setzen Sie kein Wissen als implizit vorhanden voraus, sondern sprechen Sie explizit alle Probleme\index{Probleme} und wichtigen Fakten an.
\end{itemize}
Bedenken Sie dabei stets, dass ein Leser nicht dasselbe Wissen besitzen kann wie Sie und das Sie ihm deshalb ihre Ergebnisse erklären müssen.


\subsection{Evaluation des (wissenschaftlichen) Beitrags}\index{Evaluation}
%%
Natur- bzw. ingenieurwissenschaftliche Forschung erzielt oft quantitative Ergebnisse, deren Qualität objektiv beurteilt werden muss.\index{Qualität}
Dies erfolgt üblicherweise mit Hilfe einer speziellen Evaluation, d.h. die Qualität der erzielten Ergebnisse muss mit den Ergebnissen anderer Arbeiten auf objektive Weise verglichen werden können.

\begin{itemize}
\item Oft existieren zu diesem Zweck Benchmarks\index{Benchmark}, die aber auch selbst, angepasst an die eigene, spezielle Problemstellung zusammengestellt werden können.
\item Wird ein Benchmark bzw. ein Evaluationsverfahren selbst erstellt, sollten die Autoren diesen öffentlich zur Verfügung stellen, damit die erzielten Ergebnisse nachvollziehbar werden.

{\bf Merke: Was man nicht nachvollziehen kann, wird angezweifelt.}
\item Wird ein existierender Benchmark verwendet, muss dieser korrekt zitiert werden und eigene Ergebnisse mit bereits bekannten Ergebnissen in Relation gestellt werden.
\end{itemize}



\subsection{Diskussion der Evaluationsergebnisse}
%%
Wurde eine Evaluation der erzielten Ergebnisse durchgeführt, müssen diese diskutiert werden.
Dabei sollten (falls jeweils zutreffend) folgende Fragen beantwortet werden:\index{Evaluation}
\begin{itemize}
\item Warum ist das eigene Ergebnis besser/schlechter als das zum Vergleich herangezogene?
\item Sind die erzielten Ergebnisse objektiv oder gibt es Gründe, diese eventuell in Zweifel zu ziehen?
\item Welche Vorbedingungen müssten verändert werden, um bessere Ergebnisse bzw. eine objektivere Evaluation zu erzielen?
\item Falls die Evaluation für unseren speziellen Fall nicht aussagekräftig (genug) ist, wie sollte diese verändert werden?
\end{itemize}


\subsection{Zusammenfassung und Ausblick}\index{Zusammenfassung}

In diesem Abschnitt sollten die erzielten Ergebnisse noch einmal kurz zusammengefasst werden und ein Ausblick auf weiterführende Forschungs- und Entwicklungsarbeiten gegeben werden (vgl. Kap.~\ref{sec_conclusion}).