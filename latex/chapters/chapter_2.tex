\section{Logic of Events}
\label{sec_logic}

% TODO
% Aussagenlogik +
% Prädikatenlogik 1. Ordnung +
% intuitionistische Logik +
% Martin Löf Typentheorie
% Automatentheorie NEA?
% Eventlogik +?
% RAFT Zensusprotokoll

This chapter introduces the logical and programmatic foundations needed for this
thesis. To achive this, it is split into two parts. The first part introduces
the reader with the Logic of Events which is the theoretical foundation for the
Velisarios framework. The second part provides a practical introduction to
the basics of COQ and programming with Velisarios.

\subsection{Logic}

The Logic of Events was introduced by Bickford and
Constable~\cite{bickford2003logic} to describe parallel systems which
consists out of nodes and events happening between them. To follow
the explenation a brief overview about propositional logic, predicativ logic,
constructive logic and type theorie is presented.

\paragraph{Propositional Logic}
The oldest known logic uses atoms and logical connective to formalize
statements about the world. Statements in classical logic can be either
\textbf{true} or \textbf{false}. The table~\ref{tab:proplogic} provides
a short description for the logical connectives and their meaning.

\begin{table}[h]
  \centering
  \begin{tabular}{c|c}
    Connective & Description\\\hline
    $A,B,C,...$ & Atoms used in statements\\
    $\neg$ & Negation of some atom or statement\\
    $\vee$ & \\
    $\wedge$ & \\
    $\Rightarrow$ & \\
    $\Leftrightarrow$ & \\
  \end{tabular}
  \caption{Logical connectives}
  \label{tab:proplogic}
\end{table}


% %% einleitung %%
% In diesem Kapitel wird die Eventlogik~\cite{bickford2003logic} nach Bickford und
% Constable vorgestellt. Dazu werden vorher die benötigten grundlegenden Logiken
% und Theorien eingeführt. Zuerst wird kurz die Aussagenlogik erklärt und die
% darauf aufbauende Prädikatenlogik, welche grundlegend sind um ein Verständnis
% für die späteren Formeln zu liefern. Danach wird auf die intuitionistische
% Interpretationsweise und konstruktive Logik eingegangen. Als weitere Grundlagen
% werden ein Einführung in die Typentheorie und Automatentheorie gegeben, soweit
% dass für das weitere Verständnis der Eventlogik notwendig ist.

% \subsection{Grundlegende Logiken}
% Die Mathematische Logik ist eine mathematische Theorie in der andere
% mathematische Modell mit Hilfe "`naiver Logik"' formalisiert werden können.
% Die mathematische Logik bietet damit eine Metasprache zur Untersuchung anderer
% mathematischer Modelle.~\cite{heinemann2013logik}

\paragraph{Aussagenlogik}
Sie ist die älteste Logik und benutzt aus Grundelement Aussagen (Atome,
Literale) zur Formalisierung. Aussagen sind können entweder \textbf{wahr}
oder \textbf{falsch} sein. \textbf{Terme} sind aussagenlogische Formeln
bei der einzelne Aussagen über \textbf{Junktoren} verbunden sind.~\cite{heinemann2013logik}

\begin{defi}
  Sei $\Sigma_A := \{V,(,),\neg,\vee,\wedge,\Rightarrow,\Leftrightarrow,\bot,\top\}$ das Alphabet der Aussagenlogik, wobei $\top$ für
wahr, $\bot$ für falsch, $\neg$ für die Negation, $\vee$ für die Disjunktion, $\wedge$ für die
Konjunktion , $\Rightarrow$ für die Implikation und $\Leftrightarrow$ für die Äquivalenz steht.
\begin{itemize}
\item $V:=\{p_0,p_1,..\}$  ist die Menge der Variablen oder Symbole
\item $T\subseteq W(\Sigma)$  ist die Menge der Terme die über $\Sigma$ gebildet werden können 
\end{itemize}
\end{defi}

\begin{defi}
  Ein Term ist definiert als
  \begin{itemize}
  \item $V\subseteq T$
  \item $\{\top,\bot\}\subseteq T$
  \item $\{\neg\alpha,(\alpha\vee\beta),(\alpha\wedge\beta),(\alpha\Rightarrow\beta),(\alpha\Leftrightarrow\beta)\}\subseteq T$, falls $\alpha,\beta\in T$
  \end{itemize}
\end{defi}

Als Beispiel kann der Satz "`Wenn X keine Zahl ist, dann kommt danach ein
Buchstabe."' in die Formel
\[
  \neg x_{Z} \Rightarrow x_{B}
\]
übersetzt werden.

Terme können ausgewertet werden und ihnen Wahrheitswerte zugeordnet werden.
Dazu werden Interpretationen für die vorkommenden Variablen angegeben.

\begin{defi}
  Die Interpretation ist eine mögliche Belegung für einen Term $t$ und
  wird als $I:L(t)\rightarrow \{wahr,falsch\} \Leftrightarrow t^I$ geschrieben.
  Eine Interpretation die eine Formel $t^I$ wahr werden lässt, \textbf{erfüllt}
  diese Formel.
\end{defi}

Eine besondere Art eines Terms ist die \texttt{Tautologie} bei der jegliche
Interpretion wahr wird. Eine Tautologie ist eine allgemeingültige Ausage.

\begin{defi}
  Axiome sind Terme einer Logik, die ohne Angabe von Rechtfertigungen als wahr angenommen
  werden. Axiome werden zumeist als semantische Tautologien gewählt, wobei sie
  syntaktisch gesehen formale Terme ohne besondere Bedeutung sind.
\end{defi}

Aus den Tautologien und Axiomen einer Logik können weitere Terme abgeleitet
werden. Zur Ableitung weiterer gültiger Terme aus einer Logik werden
Schlussregeln oder Inferenzregel benutzt. Ein System aus Inferenzregeln zum
Schließen über Tautologien und möglicherweise Axiomen einer Logik wird Kalkül genannt.

\begin{defi}
  Sei $\Gamma$ eine endliche Menge von Termen einer Logik und $C$ ein Term der Logik.
  Dann ist $C$ aus $\Gamma$ ableitbar, wenn es eine Folge von Inferenzregeln
  $r_0,...,r_n$ und Termen $C_0,...,C_n$ gibt, so dass $C_n = C$ und
  $\Gamma_0\vdash_{r_0}C_0 ... \Gamma_n\vdash_{r_n}C_n$ wobei $\Gamma_i$ die Menge aller Axiome, Terme
  aus $\Gamma$ und bisher abgeleiteten Terme enthält.
\end{defi}

\subsection{Prädikatenlogik}
Die Prädikatenlogik stellt eine Erweiterung der Aussagenlogik dar.
Bei der Aussagenlogik werden atomare Aussagen oder Literale mit konkreten
Wahrheitswerten belegt. In der Aussagenlogik können nicht alle Aussagen
mit Wahrheitswerten belegt werden, da manche Belegungen von der inneren
Struktur der Aussage abhängen. Zum Beispiel der Satz "`x ist eine Primzahl"'
hängt von einer konkreten Belegung $x$ ab und ist in der Aussagenlogik kein
valides Literal. Diese Abhängigkeit wird durch Prädikate ausgedrückt.~\cite{heinemann2013logik}

\begin{defi}
  Das Alphabet der Prädikatenlogik ist $\Sigma_P := \{\exists,\forall,F,V,P\} \cup \Sigma_A$, wobei
  $\exists$ als \texttt{es existiert} und $\forall$ als \texttt{für alle gilt} gelesen wird.
  $V$ ist die Menge der Variablen, $P$ ist die Menge der Prädikate und $F$ ist die Menge der
  Funktionen in der Prädikatenlogik.
\end{defi}

\begin{defi}
  Ein \textbf{Prädikat} der Stelligkeit \textbf{n} ordnet einem Variablentupel
  $(x_1,...,x_n)$ der Mengen $M_1,...,M_n$ Wahrheitswerte zu, daraus folgt
  $P(x_1,...,x_n) \rightarrow \{wahr, falsch\}$ wobei $x_i\in M_i$ für $i\in \{1,...,n\}$.
\end{defi}
Ein Prädikat, dass ein Tupel erfüllt, ordnet dem Tupel den Wahrheitswert wahr zu.

\begin{defi}
  Sei $\mathbb{F}=\Cup^{\infty}_{i=0}\mathbb{F}^i$ für jedes $0\le i$ ein Alphabet von
  i-stelligen Funktionssymbolen und $\mathbb{V}$ ein Alphabet von Variablen,
  dann sind Terme der Prädikatenlogik definiert als:
  \begin{itemize}
  \item $x\in\mathbb{V}$ ist ein Term
  \item $f\in\mathbb{F}^0$ ist ein 0-stelliges Funktionssymbole und eine Term der
    eine Konstante ist
  \item $t_1,...,t_n$ sind Terme und $f\in\mathbb{F}^n$ ein n-stelliges
    Funktionssymbol, dann ist $f(t_1,...,t_n)$ ein Term
  \end{itemize}  
\end{defi}~\cite{kreitz1994automatisierte}

Zur Darstellung von konkreten Sprachen müssen Bezeichner aus der Menge aller
Bezeichner aussortiert werden. Damit wird eine konkrete Sprache mit einem Typ
versehen.
\begin{defi}
  Ein Typ $T$ ist eine Paar $(I,J)$ mit $I\subseteq P$ und $J\subseteq F$.
\end{defi}~\cite{heinemann2013logik}

Damit ist es möglich die Formeln der Prädikatenlogik zu definieren.

\begin{defi}
  Sei $\mathbb{P}^i$ mit $0\le i$ ein i-stelliges Prädikatenalphabet,
  $\mathbb{P}=\Cup^{\infty}_{i=0}\mathbb{P}^i$ und $\mathbb{T}$ ein Alphabet von Typen,
  dann ist eine Formel der Prädikatenlogik definiert als:
  \begin{itemize}
  \item $P\in\mathbb{P}^0$ ist ein 0-stelliges Prädikatensymbol und damit P eine
    atomare Formel oder Aussagenvariable
  \item $t_1$ und $t_2$ sind Terme, dann ist $t_1 = t_2$ eine atomare Formel,
    die gewöhnliche Gleichheit ausdrückt.
  \item $t_1,...,t_n$ mit $n\geq1$ sind Terme und $P\in\mathbb{P}^n$ ein n-stelliges
    Prädikatensymbol, dann ist $P(t_1,...,t_n)$ eine atomare Formel
  \item $A,B$ sind Formel, $x\in\mathbb{V}$ eine Variable udn $T\in\mathbb{T}$ ein
    Typenbezeichner, dann sind $\neg A$, $A\vee B$, $A\wedge B$, $A\Rightarrow B$, $\forall x:T.A$, $\exists
    x:T.A$ und $(A)$ Formeln
  \end{itemize}
\end{defi}~\cite{kreitz1994automatisierte}

Eine Interpretation für eine prädikatenlogische Formel ist lediglich eine
Erweiterung der aussagenlogischen Interpretation, um die Belegungen von
n-stelligen Prädikat- und Funktionssymbolen.~\cite{heinemann2013logik}

\paragraph{Bindung}
Zur besseren Lesbarkeit werden Klammer für die Bindungsstärke von Junktoren
zumeist weggelassen. Dadurch ergibt sich die implizite Bindungsreihenfolge:
$\neg,\wedge,\vee,\Rightarrow,\exists,\forall$.~\cite{kreitz1994automatisierte}

Die Prädikatenlogik stellt das Grundgerüst, auf dessen Basis die später
vorgestellte Eventlogik steht. Sie ermöglicht die Formalisierung von natürlichen
Aussagen und Theorien, sowie die Überprüfung derer Gültigkeit.


\subsection{intuitionistische Logik}
Bisher wurden die Logik mit ihrer klassischen Interpretation vorgestellt.
Dabei wird angenommen das eine Aussage entweder wahr oder falsch ist.
Anders ausgedrückt kann ein mathematische Objekt entweder existieren oder
nicht. Beweise auf Grundlage dieser Interpretationsansicht erfordern keine
konstruktive Herleitung, wie ein solches Objekt zu modellieren ist.
Das Gesetz vom ausgesschlossenen Dritten $A\vee\neg A$, die doppelte Negation $\neg\neg A\Rightarrow
A$, Wiederspruchsbeweise und "`ex falso sequitur quodlibet"' gehören zu den
problematischen Beweisansätzen, die keine intrinsische Rechtfertigung
benötigen.~\cite{sep-logic-intuitionistic}

Brouwer hat 1908 festegestellt, dass $A\vee\neg A$ angewendet auf unendliche
abzählbare Mengen zu Problemen führt. Ein Beispiel dafür ist die Goldbach
Vermutung, die annimmt, dass jede Zahl $> 2$ als Summe zweier Primzahlen
geschrieben werden kann. Diese Aussage ist bis heute unbewiesen. Die klassische
Logik umschifft solche Probleme in dem sie annimmt, dass weder die Aussage, noch 
ihr Gegenteil gleichzeitig falsch sein kann, ohne eine konstruktive
Rechtfertigung dafür zu liefern. Dabei ist es für den geneeigten Leser
einsichtig, dass keine einfache Funktion, nach der Church-Turing These, 
gefunden werden kann, die in absehbarer Zeit eine Lösung oder Beweis der
Aussage liefert.~\cite{sep-logic-intuitionistic, sep-mathematics-constructive}

Der Intuitionismus und die konstruktive Mathematik lehnen diese Art der
Beweisführung und die Allgemeingültigkeit der vorher erwähnten Gesetze ab.
Dabei werden verschiedene logische Operationen anders interpretiert.
Zum Beispiel wird von der Disjunktion $A\vee B$ verlangt, dass sie angeben kann
welcher Teil beweisbar ist und einen entsprechenden Beweis (Evidenz)
dafür liefert.~\cite{kreitz1994automatisierte}

\begin{defi}
  Die intuitionistische Interpretation der Prädikatenlogik setzt sich aus
  Evidenzen zusammen die in den Teilaussagen gefordert sind.
  \begin{itemize}
  \item Jedes Literal $q$ besitzt eine Evidenz $e$
  \item Um $P\vee Q$ zu beweisen, ist entweder die Evidenz für $P$ oder $Q$
    notwendig
  \item Um $P\wedge Q$ zu beweisen, ist die Evidenz von $P$ und $Q$ notwendig
  \item Um $P\Rightarrow Q$ zu beweisen, muss ein Algorithmus angegeben werden, der jede
    Evidenz für $P$ in eine Evidenz für $Q$ umwandelt 
  \item Um $\neg P$ zu beweisen, muss gezeigt werden, dass es keinen Beweis gibt
  \item Um $\exists x:T.P(x)$ zu beweisen, muss ein Objekt vom Typ $T$ konstruiert
    und gezeigt werden, dass $P$ hält.
  \item Um $\forall x\in T.P(x)$ muss gezeigt werden, dass jedes Objekt $x$ den Typ $T$
    hat und darunter $P$ beweisbar ist.
  \end{itemize}
\end{defi}~\cite{sep-mathematics-constructive}

Die intuitionistische Logik ermöglicht damit Kalküle zu entwickeln auf
deren Grundlage beweisbare logische Schlüsse erzeugt werden können.
Damit kann sie als eine höhere Programmiersprache angesehen werden.
Beweise die konstruiert werden können mit Hilfe der Curry-Howard Isomorphie
in Typsysteme überführt werden und ausprogrammiert werden.~\cite{constable1970constructive}


\subsection{Martin-Löf Typentheorie}
Verschiedene Kalküle wurden für die Prädikatenlogik entwickelt.
In dieser Arbeit wird nur auf die Typentheorie von Per Martin Löf eingegangen,
da sie die Grundlage für den interaktiven Theorembeweiser
NuPRL\footnote{\url{http://www.nuprl.org/Intro/intro.html}} ist. In NuPRL ist
die später vorgestellte Eventlogik formalisiert und bewiesen worden.



%% eventlogik %%
\subsection{Eventlogik}

Mark Bickford und Richard L. Constable haben in verschiedenen
Papern~\cite{bickford2003logic, bickford2005causal, bickford2009component} eine Logik zur Beschreibung
von Ereignissen in verteilten Systemen vorgestellt. Die Eventlogik ist so
aufgebaut, dass eine breite Menge von Systemen beschrieben werden kann, auch
außerhalb des eigentlichen Anwendungsgebietes des verteilten Rechnes
(Distributed Computing).  Die Logik ist auf der
intuitionistischen Logik aufgebaut und folgt dem ``correct-by-construction''
Ansatz. Bei diesem Ansatz werden aus Beweisen für Formeln korrekte Implementierungen für
diese gewonnen.~\cite{bates1985proofs}

Dafür wird eine abstrakte Spezifikationssprache eingeführt welche durch
ein Berechenbarkeitsmodell repräsentiert wird.
Schlussfolgerungen in diesem Modell werden durch Inferenzregeln dargestellt und
ermöglichen es, wenn eine Spezifikation erfüllbar ist, ein ausführbares
verteiltes System aus dem Modell zu extrahieren.~\cite{bickford2005causal}


\paragraph{typentheoretische Vorbetrachtung}
Um der Eventlogik eine mathematische Struktur zu geben, werden ihre
Elemente mit diskreten Typen beschrieben, d.h. dass ihre die Gleichheit
entscheidbar ist und sie voneinander disjunkt sind.

\begin{defi}
  $\mathbb{D}$ ist der Typ der alle diskreten Typen enthält $T\in\mathbb{D}$ mit $\{T:Type|\forall x,y : T.x = x\ in\ T \vee\neg (x = y\ in\
  T)\}$.
\end{defi}

\subsubsection{Events}
Events bilden die Grundbausteine der Eventlogik.
Sie stellen Aktionen dar die in Raum und Zeit passieren und
werden ohne Zeitdauer definiert.
Die Zeitdauer eines Events würde sich auf die physische Zeit der
Umsetzung beziehen und wurde der zur Vereinfachung
wegabstrahiert.~\cite{bickford2005causal}

\begin{defi}
  Ein Event $e$ ist die atomare Einheit und kausal geordnet, d.h.
  $e< e'$ wenn $e$ zeitlich vor $e''$ passiert ist.
\end{defi}

\begin{defi}
  Events $e\in E$ sind in einem Eventraum strukturiert. Dieser besteht
  aus einzelnen $loci$ oder Entitäten an denen Events passieren.
\end{defi}

Der Eventraum ist dynamisch aufgebaut, so dass über die Zeit neue Entitäten
hinzugefügt oder entfernt werden können. Damit die Entitäten im Raum
unterschieden werden können besitzen sie beobachtbare Eigenschaften, wie zum
Beispiel Koordinaten oder eine ID. Zur theoretischen Betrachtung reicht es aus,
wenn Entitäten durch einen diskreten Typen und eine ID unterschieden werden können.~\cite{bickford2005causal}

\begin{defi}
  Jede Entität ist mit jeder anderen Entität über mindestens
  einen Kommunikationsweg ($link$) verbunden.
\end{defi}

Für die $links$ wird das nicht-byzantinische Fehlermodell angenommen,
wobei die Verbindungen unzuverlässig sind, aber die Nachrichten nicht
verfälschen. Ein Event kann zur Kommunikation Nachrichten über einen
$link$ versenden.

Da Events an Entitäten gebunden sind, können sie kausal strukturiert werden.
Damit können Events folgendermaßen formalisiert werden.
\begin{gather*}
  \textbf{E:}\mathbb{D}\notag\\
  \textbf{Loc:}\mathbb{D}\\
  \textbf{pred?:} E\rightarrow (E+Loc)\\
  \textbf{sender?:} E\rightarrow (E+Unit)
\end{gather*}

E ist ein Event und Loc eine Entität. Die Funktion $pred?$ gibt entweder das
vorherige Event an der Entität zurück oder die Entität, wenn es das erste Event
war. Die Funktion $sender?$ gibt des Event $e'$ zurück, dass das Event $e$
ausgelöst hat oder den leeren Typ.~\cite{bickford2005causal}

Mit diesen Grundannahmen können die Funktionen $first?$, die bestimmt ob
ein Event das Erste ist, dass bei einer Entität passiert und $recv?$,
dass bestimmt ob ein Event ein Empfangsevent ist, definiert werden.

\begin{gather*}
  \textbf{first?:}E\rightarrow (E\rightarrow (E+Loc))\rightarrow B\\
  first?(e,pred?) = if\ is\_left(pred?(e))\ then\ true\ else\ false\\
  \textbf{recv?:}E\rightarrow (E\rightarrow (E+Unit))\rightarrow B\\
  recv?(e,sender?) = if\ is\_left(sender?(e))\ then\ true\ else\ false\\
\end{gather*}

Mit Hilfe dieser Funktionen lässt sich die Eingangserwähnte kausale Ordnung auf Events
$e < e'$ definieren als:
\[
  pred!(e,e') == (\neg first(e')\Rightarrow e = pred?(e')) \vee e = sender(e')
\]
$pred!$ bildet eine transitive Hülle und stellt damit eine wohlgeordnete
entscheidbare Ordnung der Form $e < e'$ dar. Als letztes fehlen noch
drei Grundaxiome für den geordneten Eventraum.~\cite{bickford2005causal}

\begin{axiom}
  Wenn ein Event $e$ ein Signal sendet, dann gibt es ein Event $e'$, so dass für
  jedes Event $e''$ gilt, $e'' = e'$ oder $e'' < e'$.\\
  \[
    \forall e:E \exists e':E. \forall e'':E . (recv?(e'') \& sender?(e'')=e)\Rightarrow (e' = e' \vee e'' < e)
  \]
\end{axiom}

\begin{axiom}
  Die Funktion $pred?$ ist injektiv.
  \[
    \forall e,e':E.loc(e) = loc(e')\Rightarrow pred?(e) = pred?(e')\Rightarrow e=e'
  \]
\end{axiom}

\begin{axiom}
  Die Funktion $pred!$ bildet eine starke wohlfundierte Ordnung.
  \[
    \exists f:E\rightarrow \mathbb{N}.\forall e,e':E.pred!(e,e')\rightarrow f(e)<f(e')
  \]
\end{axiom}

\begin{defi}
  Die folgenden Notationen werden als Abkürzung verwendet, um Events die an einem
  Ort stattfinden zu quantifizieren.
  \[
    \forall e@i.P == \forall e:E.(loc(e) = i\Rightarrow P)
  \]
  \[
    \exists e@i.P == \exists e:E.(loc(e) = i\Rightarrow P)
  \]
\end{defi}

\begin{figure}
  \center
  \begin{tikzpicture}[node distance=2cm,auto,>=stealth]
    \node[] (server) {$R$};
    \node[left = of server] (client) {$S$};
    \node[below of=server, node distance=3cm] (server_ground) {};
    \node[below of=client, node distance=3cm] (client_ground) {};
    % vertical line
    \draw (client) -- (client_ground);
    \draw (server) -- (server_ground);
    % horizontal lines
    \draw[->] ($(client)!0.2!(client_ground)$) -- node[above,scale=1,midway]{$m_1$} ($(server)!0.3!(server_ground)$);
    \draw[<-] ($(client)!0.5!(client_ground)$) -- node[below,scale=1,midway]{$m_2$} ($(server)!0.4!(server_ground)$);
    \draw[->] ($(client)!0.6!(client_ground)$) -- node[below,scale=1,midway]{$m_3$} ($(server)!0.7!(server_ground)$);
    \draw[<-] ($(client)!0.9!(client_ground)$) -- node[below,scale=1]{$m_4$} ($(server)!0.8!(server_ground)$);
  \end{tikzpicture}
  \label{fig:sequence}
  \caption{Ein einfaches Sequenzdiagramm zwischen zwei Knoten ($S,R$) die jeweils
    zwei Nachrichten ($m_i$) übertragen.}
\end{figure}

Abbildung~\ref{fig:sequence} zeigt ein einfaches Beispiel über einen Eventraum,
der sich mit der bisherigen Theorie spezifizieren lässt. Dort wird eine 1:1
Kommunikation zwischen zwei Entitäten dargestellt. Jede Nachricht von S wird mit
einer Antwort von R quitiert, bevor eine neue Nachricht von S bei R eintrifft.
Diese Modell kann forlgendermaßen formalisiert werden:~\cite{bickford2005causal}

\begin{gather*}
  \forall s@S.\exists r'@R.s=sender?(r')\\
  \exists r@S.(s<r\&\exists s''@R.(r'\leq s''\& sender?(r) == s''))\\
%%  \forall x@S.s<x<r\Rightarrow x wird nicht an R gesendet\\
  \forall x@S.s<x<r\Rightarrow \neg\exists x'@R.(x=sender?(x'))
\end{gather*}

Als nächstes werden Events erweitert um Werte und unterschieden nach
$internen$ und $externen$ Events. Bei einem $externes$ Event sendet ein Sender
einer Nachricht über einen $link$ zu einem Empfänger. $Interne$ Events können
ebenfalls Werte übergeben und können durch $guards$ limitiert werden.~\cite{bickford2005causal}

\begin{defi}
  Der Typ eines Events ist:
  \[
    kind == (Act+Top)
  \]
  \[
    kind: E\rightarrow (Act+Top)
  \]
  Dabei ist Act ein diskreter Typ um unterschiedle interne Events, zu
  unterscheiden und Top der Wert eines externen Events.
\end{defi}

Eine Darstellungsform von Interaktionen und Eventräumen sind Sequenzdiagramme
mit Nachrichten, wie in Abb.~\ref{fig:sequence} dargestellt.
% EValues vllt vertiefen, funktionalität beschreiben?


\begin{defi}
  Eine Entität kann einen Zustand haben der durch eine endliche Folge
  von Änderungen dargestellt werden kann. $s'=f(s,v)$, wobei
  $v'$ der Folgezustand und $v$ der Wert eines Events ist.  
\end{defi}


% Die Eventlogik nach Bickford und Constable baut auf der intuitionistischen Logik
% auf. Sie verfolgt den ``correct-by-construction'' Ansatz, nach dem ein
% funktionales Programm aus einem konstruktiven Beweis extrahiert werden kann.
% Die Beweise haben die Grundform $\forall x:A.\exists : B.spec(x,y)$. Der konstruktive Beweis
% liefert einen Extraktterm, der eine Realisierung des Beweises darstellt.
% Da die Eventlogik die konstruktive Typentheorie benutzt sind alle
% beweisbaren logischen Aussagen auch programmierbar, in dem Sinne, dass sie einen
% Extraktterm besitzen der die Spezifikation erfüllt.~\cite{bickford2003logic}

% nachrichtenautomat
Grundlage für die Eventlogik ist ein Nachrichtenautomat. Dieser ist ein
nichtdeterministischer Automat der auf Nachrichten operiert.
Er hat die drei Basisoperationen ``senden'', ``empfangen'' und ``interne
Zustandsänderung''.~\cite{bickford2003logic}

\begin{defi}
  Ein Event $e=(a+m)$ ist ein Typ $A+M$ der entweder eine interne Aktion $A$ enthält
  oder eine empfangene Nachrichten $M$.
\end{defi}

\begin{defi}
  Ein Nachrichtenautomat ist ein abhängiger Datentyp mit der Struktur:
  $\{St, Act, Msg: Type, init: St$
    $f:(Act+Msg)\rightarrow St\rightarrow St$
    $send:(Act+Msg)\rightarrow St\rightarrow MsgList\}$  
\end{defi}

% verteilte Nachrichtenautomaten
Um verteilte Systeme darstellen zu können wird mehr als ein Nachrichtenautomat
benötigt, da ein Nachrichtenautomatenohne Interaktionen ein normaler
nichtdeterministischer Automat ist. Ein verteiltes System $V$ ist eine Menge von
Nachrichtenautomaten die in einem Netzwerk miteinander verknüpft sind.
Das Netzwerk ist fair, d.h. jede gesendete Nachricht wird auch unverändert
empfangen.~\cite{bickford2003logic}

\begin{defi}
  Ein verteiltes System $V$ ist ein Tupel $V=(M,L)$ wobei $m_i\in M$ mit $i\in
  \mathbb{N}$ die Menge der teilnehmenden Nachritenautomaten ist. $L$ beschreibt
  ein Kante in einem gerichteten Graphen und besteht aus dem Tupel $L=(s,d,m_i,m_j,Q)$ 
  mit $i,j\in \mathbb{N} und i!= j.Q$ ist eine Nachrichtenqueue.
\end{defi}

%%% Local Variables:
%%% mode: latex
%%% TeX-master: "../master"
%%% End:
