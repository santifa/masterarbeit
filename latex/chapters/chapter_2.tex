%%% Local Variables:
%%% mode: latex
%%% TeX-master: "../master"
%%% End:

\section{Eventlogik}
\label{sec_logik}

%% einleitung %%
In diesem Kapitel wird die Eventlogik~\cite{bickford2003logic} nach Bickford und
Constable vorgestellt. Dazu werden die benötigten grundlegenden Logiken
und Theorien eingeführt.

%% grundlagen %%
\subsection{Grundlegende Logiken}
% TODO
% Aussagenlogik
% Prädikatenlogik
% Logik 1. Ordnung
% dynamische Logik?
% Automatentheorie NEA

%% eventlogik %%
\subsection{Eventlogik}

Mark Bickford und Richard L. Constable haben in verschiedenen
Papern~\cite{bickford2003logic, bickford2005causal, bickford2009component} eine Logik zur Beschreibung
von Ereignissen in verteilten Systemen vorgestellt. Die Eventlogik ist so
aufgebaut, dass eine breite Menge von Systemen beschrieben werden kann, auch
außerhalb des eigentlichen Anwendungsgebietes des verteilten Rechnes
(Distributed Computing).  Die Logik ist auf der
intuitunistischen Logik aufgebaut und folgt dem ``correct-by-construction''
Ansatz. Bei diesem Ansatz werden aus Beweisen für Formeln korrekte Implementierungen für
diese gewonnen.~\cite{bates1985proofs}

Dafür wird eine abstrakte Spezifikationssprache eingeführt welche durch
ein Berechenbarkeitsmodell repräsentiert wird.
Schlussfolgerungen in diesem Modell werden durch Inferenzregeln dargestellt und
ermöglichen es, wenn eine Spezifikation erfüllbar ist, ein ausführbares
verteiltes System aus dem Modell zu extrahieren.~\cite{bickford2005causal}

\subsubsection{Events}
Events bilden die Grundbausteine der Eventlogik.
Sie stellen Aktionen dar die in Raum und Zeit passieren und
werden ohne Zeitdauer definiert.~\cite{bickford2005causal}

\begin{defi}
  Ein Event $e$ ist die atomare Einheit und kausal geordnet, d.h.
  $e\le e'$ wenn $e$ zeitlich vor $e'$ passiert ist.
\end{defi}

Events sind in einem Eventraum organisiert, der aus diskreten Orten ($loci$)
oder Entitäten besteht. Events passieren an den $loci$ und dem Eventraum
können über die Zeit beliebig neue Ort hinzugefügt oder entfernt werden.
Die verschiedenen $loci$ können beobachtbare Eigenschaften besitzen.~\cite{bickford2005causal}

\begin{defi}
  Jeder $loci$ ist mit allen anderen $loci$ uber mindestens einem
  Kommunikationskanal verbunden. Die Kommunikationskanäle müssen
  nicht zuverlässig sein, aber dürfen die Nachricht nicht verfälschen.
\end{defi}

Events versenden entweder Nachrichten zu einer anderen Entität oder
lösen interne Veränderungen aus, wobei $interne$ Events nur unter
gewissen Voraussetzungen ausgelöst werden. $Externe$ Events
können vom Empfänger nicht blockiert werden.~\cite{bickford2005causal}

\begin{defi}
  Eine Entität kann einen Zustand haben der durch eine endliche Folge
  von Änderungen dargestellt werden kann. $s'=f(s,v)$, wobei
  $v'$ der Folgezustand und $v$ der Wert eines Events ist.  
\end{defi}



\begin{figure}
  \center
  \begin{tikzpicture}[node distance=2cm,auto,>=stealth]
    \node[] (server) {$N_B$};
    \node[left = of server] (client) {$N_A$};
    \node[below of=server, node distance=5cm] (server_ground) {};
    \node[below of=client, node distance=5cm] (client_ground) {};
    % vertical line
    \draw (client) -- (client_ground);
    \draw (server) -- (server_ground);
    % horizontal lines
    \draw[->] ($(client)!0.25!(client_ground)$) -- node[above,scale=1,midway]{$m_1$} ($(server)!0.3!(server_ground)$);
    \draw[<-] ($(client)!0.45!(client_ground)$) -- node[below,scale=1,midway]{$m_2$} ($(server)!0.35!(server_ground)$);
    \draw[->] ($(client)!0.5!(client_ground)$) -- node[below,scale=1,yshift=5mm]{$m_3$} ($(server)!0.8!(server_ground)$);
    \draw[<-] ($(client)!0.9!(client_ground)$) -- node[below,scale=1]{$m_4$} ($(server)!0.6!(server_ground)$);
  \end{tikzpicture}
  \label{fig:sequence}
  \caption{Ein einfaches Sequenzdiagramm zwischen zwei Knoten ($N_{A,B}$) die jeweils
    zwei Nachrichten ($m_i$) übertragen.}
\end{figure}


% Die Eventlogik nach Bickford und Constable baut auf der intuitunistischen Logik
% auf. Sie verfolgt den ``correct-by-construction'' Ansatz, nach dem ein
% funktionales Programm aus einem konstruktiven Beweis extrahiert werden kann.
% Die Beweise haben die Grundform $\forall x:A.\exists : B.spec(x,y)$. Der konstruktive Beweis
% liefert einen Extraktterm, der eine Realisierung des Beweises darstellt.
% Da die Eventlogik die konstruktive Typentheorie benutzt sind alle
% beweisbaren logischen Aussagen auch programmierbar, in dem Sinne, dass sie einen
% Extraktterm besitzen der die Spezifikation erfüllt.~\cite{bickford2003logic}

% nachrichtenautomat
Grundlage für die Eventlogik ist ein Nachrichtenautomat. Dieser ist ein
nichtdeterministischer Automat der auf Nachrichten operiert.
Er hat die drei Basisoperationen ``senden'', ``empfangen'' und ``interne
Zustandsänderung''.~\cite{bickford2003logic}

\begin{defi}
  Ein Event $e=(a+m)$ ist ein Typ $A+M$ der entweder eine interne Aktion $A$ enthält
  oder eine empfangene Nachrichten $M$.
\end{defi}

\begin{defi}
  Ein Nachrichtenautomat ist ein abhängiger Datentyp mit der Struktur:
  $\{St, Act, Msg: Type, init: St$
    $f:(Act+Msg)\rightarrow St\rightarrow St$
    $send:(Act+Msg)\rightarrow St\rightarrow MsgList\}$  
\end{defi}

% verteilte Nachrichtenautomaten
Um verteilte Systeme darstellen zu können wird mehr als ein Nachrichtenautomat
benötigt, da ein Nachrichtenautomatenohne Interaktionen ein normaler
nichtdeterministischer Automat ist. Ein verteiltes System $V$ ist eine Menge von
Nachrichtenautomaten die in einem Netzwerk miteinander verknüpft sind.
Das Netzwerk ist fair, d.h. jede gesendete Nachricht wird auch unverändert
empfangen.~\cite{bickford2003logic}

\begin{defi}
  Ein verteiltes System $V$ ist ein Tupel $V=(M,L)$ wobei $m_i\in M$ mit $i\in
  \mathbb{N}$ die Menge der teilnehmenden Nachritenautomaten ist. $L$ beschreibt
  ein Kante in einem gerichteten Graphen und besteht aus dem Tupel $L=(s,d,m_i,m_j,Q)$ 
  mit $i,j\in \mathbb{N} und i!= j.Q$ ist eine Nachrichtenqueue.
\end{defi}
