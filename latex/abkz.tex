\section{Abkürzungen und Akronyme}\index{Akronyme}
\label{akro}
\begin{tabbing}
\hspace*{3cm}\=  \\ \kill
A2KB     \> Annotate to knowledge base\\
ASCII    \> American Standard Code for Information Interchange\\
C2KB     \> Concept to knowledge base\\
D2KB     \> Disambiguate to knowledge base\\
FTP      \> File Transfer Protocol\\
HITS     \> Hypertext-Induced Topic Search\\
HTML     \> Hypertext Markup Language\\
HTTP     \> Hypertext Transfer Protocol\\
IATA     \> International Air Transport Association\\
IE       \> Informationsextraktion\\
IEC      \> International Electrotechnical Commission\\
IRI      \> Internationalized Resource Identifiers\\
ISO      \> International Organization for Standardization\\
LOD      \> Linked Open Data\\
NED      \> Named Entity Disambigutation\\
NEL      \> Named Entity Linking\\
NEN      \> Named Entity Normalisation\\
NER      \> Named Entity Recognition\\
NIL      \> Not in List\\
NLP      \> Natural Language Processing\\
RDF      \> Resource Description Framework\\
RDFS     \> Resource Description Framework Schema\\
OWL      \> OWL 2 Web Ontology Language\\
SPARQL   \> SPARQL Protocol and RDF Query Language\\
URI      \> Uniform Resource Identifier\\
URL      \> Uniform Resource Locator\\
W3C      \> World Wide Web Community\\
WWW      \> World Wide Web\\
XML      \> Extensible Markup Language\\
\end{tabbing}
\newpage

\section{Präfixdefinitionen}\index{Präfix}
\label{präfix}

Im Verlauf dieser Arbeit werden verschiedene Internationalized Resource Identifiers in abgekürzter Form benutzt. 
Im Folgenden sind diese aufgezählt:
\begin{tabbing}
\hspace*{3cm}\=  \\ \kill
\verb|dbo:| \> \url{<http://dbpedia.org/ontology/>}\\
\verb|dbr:| \> \url{<http://dbpedia.org/resource/>}\\
\verb|rdf:| \> \url{<http://www.w3.org/1999/02/22-rdf-syntax-ns#>}\\
\verb|rdfs:| \> \url{<http://www.w3.org/2000/01/rdf-schema#>}\\
\verb|owl:| \> \url{<http://www.w3.org/2002/07/owl#>}\\
\verb|foaf:| \> \url{<http://xmlns.com/foaf/0.1/>}
\end{tabbing}
\newpage