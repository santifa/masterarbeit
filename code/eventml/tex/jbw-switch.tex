% LaTeX (and TeX) do not provide any easy way to pass parameters to a
% document from the command line.  This file sets up switches to
% easily allow formatting different versions of document from the
% command line.  The idea is that the main document main.tex begins
% with:
%
%   % Usage: 
%
%   \DeclareSwitch{XXX}
%
% If \ifXXX is undefined, this is equivalent to:
%
%   \newif\ifXXX
%   \XXXfalse
%
% Otherwise it does nothing.

% This file may be loaded many times.  On the first time, we want to
% use \newcommand.  On subsequent occasions, we want to use
% \CheckCommand.

\@ifundefined{DeclareSwitch}{\newcommand}{\CheckCommand}
  {\DeclareSwitch}[1]
  {\@ifundefined{if#1}%
    {% First, allocate the new boolean switch.
     \expandafter\newif\csname if#1\endcsname
     % Then initialize it to false.
     \csname #1false\endcsname}%
    {% We should check here that it is a proper boolean, but that is difficult.
    }%
   \message{(if#1=if\csname if#1\endcsname true\else false\fi)}}

%   \DeclareSwitch{XXX}
%
% Following this, the TeX conditional \ifXXX can be used.  It will
% default to \iffalse, if it has not already been created
% To get per-person formatting, create a file main-XXX.tex containing:
%
%   \newif\ifXXX
%   \XXXtrue
%   \input main.tex
%
% Then you can run latex on the document with \ifXXX set to \iftrue
% by:
%
%   latex main-XXX
%
% You can run latex on the document with \ifXXX set to \iffalse by:
%
%   latex main

% This can not be a package because \usepackage barfs and dies if you
% try to use it before \documentclass.  We need to load this file
% before invoking \documentclass because we often need the features of
% this file to decide what arguments to feed to \documentclass.

\makeatletter
  % Usage: 
%
%   \DeclareSwitch{XXX}
%
% If \ifXXX is undefined, this is equivalent to:
%
%   \newif\ifXXX
%   \XXXfalse
%
% Otherwise it does nothing.

% This file may be loaded many times.  On the first time, we want to
% use \newcommand.  On subsequent occasions, we want to use
% \CheckCommand.

\@ifundefined{DeclareSwitch}{\newcommand}{\CheckCommand}
  {\DeclareSwitch}[1]
  {\@ifundefined{if#1}%
    {% First, allocate the new boolean switch.
     \expandafter\newif\csname if#1\endcsname
     % Then initialize it to false.
     \csname #1false\endcsname}%
    {% We should check here that it is a proper boolean, but that is difficult.
    }%
   \message{(if#1=if\csname if#1\endcsname true\else false\fi)}}

\makeatother
